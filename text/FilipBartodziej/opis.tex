\documentclass[
12pt
]{article}
\usepackage[T1]{fontenc}
\usepackage[utf8]{inputenc}
\usepackage{amsthm}
\usepackage{amsmath} 
\usepackage{hyperref}
\usepackage{cleveref}
\usepackage{sectsty}
\usepackage{algorithmicx,algpseudocode}

\sectionfont{\fontsize{15}{15}\selectfont}

\newtheorem{theorem}{Theorem}
\newtheorem{corollary}{Corollary}[theorem]
\newtheorem{lemma}[theorem]{Lemma}

\theoremstyle{definition}
\newtheorem{definition}{Definition}

\theoremstyle{remark}
\newtheorem*{remark}{Remark}


\algnewcommand\algorithmicinput{\textbf{INPUT:}}
\algnewcommand\Input{\item[\algorithmicinput]}

\algnewcommand\algorithmicoutput{\textbf{OUTPUT:}}
\algnewcommand\Output{\item[\algorithmicoutput]}

\begin{document}

\begin{center}
\Huge Algorytm small-large
\end{center}


\section*{Tablica sufiksów}
Dla słowa s niech $T_i$ oznacza sufiks rozpoczynający się na indeksie i, a $t_i$ to znak na i-tej pozycji. Tablica sufiksów s to tablica sufiksów s (reprezentowanych przez indeksy), to tablica posortowana według porządku leksykograficznego na sufiksach. Tablica ta przydaje się np. przy budowaniu drzew sufiksowych. Poniżej przedstawiony jest algorytm budujący tablicę sufiksów w czasie i pamięci liniowej.

W poniższych rozważaniach bez straty ogólności zakładamy, że każde słowo s kończy się znakiem unikalnym i najmniejszym w słowie. Zakładamy również że alfabet to \{1,2,...,n\}.

\section*{SL - kategoryzacja sufiksów}

Na początek przedstawimy kategoryzacje prefiksów na typy S (small) i L (large). Główna idea algorytmu jest oparta właśnie na tej kategoryzacji i jej własnościach.

\begin{definition}
Sufiks i jest typu S/L, gdy jest mniejszy/większy od sufiksu i+1. Sufiks n jest S i L (zależnie od potrzeby).
\end{definition}

Można, używając poniższego faktu, w czasie i pamięci liniowej dokonać kategoryzacji sufiksów na typy S/L.

\begin{remark}
Dla sufiksu $T_i \ (i<n)$:

\begin{itemize}
\item jeśli $t_i \neq t_{i+1}$, to wystarczy porównać jeden znak aby dowiedzieć się jakiego typu jest $T_i$
\item
jeśli $t_{i} = t_{i+1}$, to znajdź pierwszy $j > i$, taki że $t_j \neq t_i$, wtedy \\
$ t_j > t_i \Rightarrow T_i,T_{i+1},...,T_{j-1}$ są typu S \\
$ t_j < t_i \Rightarrow T_i,T_{i+1},...,T_{j-1}$ są typu L
\end{itemize}

\end{remark}

Przedstawmy również ważną własność kategoryzacji SL.

\begin{lemma}\label{SL_order}
Sufiks typu S jest leksykograficznie większy od każdego sufiksu typu L, który rozpoczyna się tym samym znakiem.
\end{lemma}
\begin{proof}
Dowód nie wprost - załóżmy że istnieją sufiksy: $T_i = c \alpha c_1 \beta$ typ S i $T_j = c\alpha c_2 \gamma$ typ L, $c_1 \neq c_2$, takie że $T_i < T_j$.

\item[Case 1:] $\alpha$ zawiera znak inny niż $c$\\
Niech $c_3$ - znak $\neq c$ najbardziej na lewo w $\alpha$. $T_i$ jest typu S, więc $c_3>c$. Podobnie z $T_j$ typu L otrzymujemy $c_3<c$, sprzeczność.

\item[Case 2:] $\alpha$ zawiera tylko $c$ lub jest puste\\
Z typów $T_i$ i $T_j$ mamy $c_1 \geq c$ i $c \geq c_2$ (dowód podobnie jak case 1). Stąd $c_1 \geq c_2$, co daje sprzeczność z $T_i < T_j$.

\end{proof}

\section*{Algorytm small-large}

Na algorytm można patrzeć w dwóch fazach - sortowanie sufiksów S/L, w zależności których z nich jest mniej. Następnie wykorzystujemy otrzymane informacje utworzenie porządku na wszystkich sufiksach. Wybieranie gałęzi z mniejszą ilością prefiksów danego typu jest konieczne aby zachować złożoność, ale nie wpływa na poprawność algorytmu.


\begin{algorithmic}
\Procedure{small-large}{$s$}\Comment{compute suffix array}
\State $SL \gets SL\_categorization(s)$
\If{$SL.count("S") < len(n)/2$}
	\State $sortedS \gets sortS(s, SL)$
	\State \Return $finishS(s, SL, sortedS)$
\Else
	\State $sortedL \gets sortL(s, SL)$
	\State \Return $finishL(s, SL, sortedS)$
\EndIf
\EndProcedure
\end{algorithmic}

W kolejnych sekcjach opisane zostaną bardziej szczegółowo funkcje $sortS$ i $finishS$. Procedury dla typu L są bardzo podobne - należy przeglądać niektóre listy w odwrotnej kolejności, operacje przeniesienia na początek/koniec grupy zamienić na przeniesienie na koniec/początek i zastąpić odniesienia do typu S odniesieniami do typu L.

\section*{Sortowanie S sufiksów - faza 1}

\begin{definition}
S odległość - dystans od i do najbliższego na lewo S sufiksu (nie wliczając i).
\end{definition}

\begin{definition}
S podsłowo - podsłowo $t[i,i+1,...,j]$, gdzie i oraz j to S sufiksy, a pomiędzy są same L sufiksy.
\end{definition}

S podsłowa są ściśle powiązane z S sufiksami - gdy weźmiemy ciąg kolejnym S podsłów i skleimy je ze sobą, otrzymamy S sufiks modulo duplikaty znaków na zetknięciu podsłów.

\begin{algorithmic}
\Procedure{sortS}{s, SL}
\State $SsubstrPrefixes \gets$ S substring prefixes, divided by Sdist, bucketed and sorted by first character
\State $Ssubstr \gets$ S substrings, bucketed and sorted by first character
\For{$d \gets 1,...,max\_Sdist$}
\For{$j$ in $SsubstrPrefixes[d]$}
\State move $j-d$ to the front of its group in $Ssubstr$
\EndFor
\State update groups in $Ssubstr$ using old $Ssubstr$ and $SsubstrPrefixes[d]$ 
\EndFor
\State $s' \gets$ map $Ssubstr$ to its buckets number, ordered as in $s$
\State \Return $unmap(small-large(s'))$
\EndProcedure
\end{algorithmic}

Po pętli for S podsłowa są uporządkowane w kolejności prawie leksyko-graficznej. Jedyny wyjątek to gdy $\alpha$ jest podsłowem $\beta$, wtedy $\alpha > \beta$ w tym porządku.

Oznaczmy jako $T'_i$ sufiks w $s'$ odpowiadający S sufiksowi $T_i$. Poprawność sortowania wynika z poniższego faktu.

\begin{lemma}
$T_i < T_j \iff T'_i < T'_j$
\end{lemma}
\begin{proof}
\item[$T_i < T_j \Rightarrow T'_i < T'_j$] - trywialnie z mapowania
\item[$T_i < T_j \Leftarrow T'_i < T'_j$] - opis idei\\
Rozważa się pierwszy znak różniący $T'_i < T'_j$. Jest to numer jakiejś grupy, która indukuje S podsłowo. Odzyskujemy te podsłowa $\alpha,\beta$ i rozważamy dwa przypadki - $\beta$ to prefiks $\alpha$ lub nie. Z tego można wywnioskować $T_i < T_j$.
\end{proof}


\section*{Sortowanie całości - faza 2}
\begin{algorithmic}
\Procedure{finishS}{s, SL, sortedS}
\State $A \gets$ bucket and sort $s$ suffixes by first character
\For{$x$ in $reversed(sortedS)$}\Comment{\cref{SL_order}}
\State move $x$ to the end of its bucket in $A$
\State move that buckets end to the left by one
\EndFor
\For{$i \gets 1,...,n$ }\Comment{\cref{phase2-invariant}}
\If{$SL[A[i]-1] == "L"$}
\State move $A[i]-1$ to the front of its bucket
\State move that buckets front to the right by one
\EndIf
\EndFor
\State \Return A
\EndProcedure
\end{algorithmic}

Po pierwszej pętli sufiksy typu S są już na poprawnych pozycjach. Pętla druga porządkuje resztę.

\begin{lemma}\label{phase2-invariant}
W kroku i drugiej pętli algorytmu, sufiks $T_{A[i]}$ jest już na poprawnej pozycji w tablicy sufiksów.
\end{lemma}
\begin{proof}
Indukcja.
\item[Baza indukcji:] wynika z tego, że najmniejszy sufiks jest typu S wiec jest już na poprawnej pozycji
\item[Krok indukcyjny $i \rightarrow i+1$:] nie wprost - istnieje sufiks $k > i+1$, który w tablicy sufiksów powinien zająć miejsce $A[i+1]$.\\
Wprost z założenia $T_{A[k]} < T_{A[i+1]}$ i obydwa te sufiksy są typu L - sufiksy S są na poprawnych miejscach.\\
$T_{A[k]}$ i $T_{A[i+1]}$ muszą rozpoczynać się tą samą literą c - przez wcześniejsze dzielenie na grupy przez pierwszą literę $T_{A[i+1]}$ ma jako pierwszą najmniejszą literę z sufiksów $A[i+1],A[i+2],...,A[n]$. Niech $T_{A[i+1]}=c \alpha$ i $T_{A[k]}=c \beta$.\\
\begin{gather*}
T_{A[k]}\ is\ type\ L \Rightarrow \beta < T_{A[k]}\\
T_{A[k]} < T_{A[i+1]} \Rightarrow \beta < \alpha\\
(T_{A[k]}\ is\ A[i+1]th\ in\ suffix\ array) \land \beta < T_{A[k]} \Rightarrow \beta \in \{A[1],A[2],...,A[i]\}
\end{gather*}
$\beta < \alpha$, więc $T_{A[k]}$ był przesunięty na początek grupy (w branchy if) przed $T_{A[i+1]}$. Sufiksy te należą do tej samej grupy, zatem $T_{A[k]}$ leży na lewo od $T_{A[i+1]}$, sprzeczność.
\end{proof}

Z \cref{SL_order} i \cref{phase2-invariant} wynika poprawność algorytmu.


\section*{Złożoność}
Ze względu na rekurencję $T(n) = T(n/2) + \mathcal{O}(n)$ otrzymujemy złożoność czasową i pamięciową liniową w stosunku do długości słowa.


\end{document}