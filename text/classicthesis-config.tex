%%%%%%%%%%%%%%%%%%%%%%%%%%%%%%%%%%%%%%%%%
% Classicthesis Typographic Thesis
% Configuration File
%
% Original author:
% André Miede (http://www.miede.de) with extensive commenting changes by:
% Vel (vel@LaTeXTemplates.com)
%
% License:
% GNU General Public License (v2)
%%%%%%%%%%%%%%%%%%%%%%%%%%%%%%%%%%%%%%%%%

\PassOptionsToPackage{utf8}{inputenc}
\usepackage{inputenc}

\PassOptionsToPackage{pdfspacing,subfig,beramono,eulermath,tocaligned,eulerchapternumbers,manychapters}{classicthesis}

\newcommand{\myTitle}{Algorytmy tekstowe\xspace}
\newcommand{\mySubtitle}{\xspace}
\newcommand{\myDegree}{dr inż.\xspace}
\newcommand{\myName}{Krzysztof Turowski\xspace}
\newcommand{\myFaculty}{Wydział Matematyki i Informatyki\xspace}
\newcommand{\myDepartment}{Instytut Informatyki Analitycznej\xspace}
\newcommand{\myUni}{Uniwersytet Jagielloński\xspace}
\newcommand{\myLocation}{Gdańsk-Kraków\xspace}
\newcommand{\myTime}{Semestr letni 2019/2020\xspace}
\newcommand{\myVersion}{wersja 1.0\xspace}

\newcounter{dummy}

\usepackage{classicthesis}
\usepackage[
    type={CC},
    modifier={by},
    version={4.0},
    lang={Polish},
    imagedistance={5em},
    imagemodifier={-eu-80x15}
]{doclicense}

\areaset[current]{411pt}{761pt}
\setlength{\marginparwidth}{72pt}
\setlength{\marginparsep}{0em}
\setlength{\oddsidemargin}{20pt}
\setlength{\evensidemargin}{0pt}
\setlength{\hoffset}{11pt}

\usepackage{csquotes}
\PassOptionsToPackage{
backend=bibtex8,bibencoding=ascii,language=auto,
style=authoryear-comp,
bibstyle=authoryear,dashed=true
sorting=ynt,
maxbibnames=10,
natbib=true
}{biblatex}
\usepackage{biblatex}

\PassOptionsToPackage{fleqn}{amsmath}
\usepackage{amsmath,amsthm,amsfonts,amssymb}
\PassOptionsToPackage{polish}{babel}
\usepackage{babel}
\PassOptionsToPackage{T1}{fontenc}
\usepackage{fontenc}
\usepackage{textcomp}
\usepackage{xspace}
\usepackage{lipsum}
\usepackage{todonotes}
\usepackage{mathtools}
\usepackage{enumitem}
\setlist{nosep}

\PassOptionsToPackage{pdftex}{graphicx}
\usepackage{graphicx}
\usepackage{tabularx}
\setlength{\extrarowheight}{3pt}
\newcommand{\tableheadline}[1]{\multicolumn{1}{c}{\spacedlowsmallcaps{#1}}}
\newcommand{\myfloatalign}{\centering}
\usepackage{caption}
\captionsetup{font=small}
\usepackage{subfig}

\PassOptionsToPackage{pdftex,hyperfootnotes=false,pdfpagelabels}{hyperref}
\usepackage{hyperref}
\pdfcompresslevel=9
\pdfadjustspacing=1
\hypersetup{
colorlinks=true, linktocpage=true, pdfstartpage=3, pdfstartview=FitV,
%colorlinks=false, linktocpage=false, pdfborder={0 0 0}, pdfstartpage=3, pdfstartview=FitV, 
breaklinks=true, pdfpagemode=UseNone, pageanchor=true, pdfpagemode=UseOutlines,%
plainpages=false, bookmarksnumbered, bookmarksopen=true, bookmarksopenlevel=1,%
hypertexnames=true, pdfhighlight=/O,%nesting=true,
frenchlinks,%
urlcolor=webbrown, linkcolor=RoyalBlue, citecolor=webgreen, %pagecolor=RoyalBlue,%
    %urlcolor=Black, linkcolor=Black, citecolor=Black, %pagecolor=Black,%
%------------------------------------------------
% PDF file meta-information
pdftitle={\myTitle},
pdfauthor={\textcopyright\ \myName},
pdfsubject={},
pdfkeywords={},
pdfcreator={pdfLaTeX},
pdfproducer={LaTeX classicthesis}
}
\usepackage{cleveref}

\theoremstyle{plain}
\newtheorem{theorem-thm}{Twierdzenie}[section]
\newtheorem{lemma-thm}[theorem-thm]{Lemat}
\theoremstyle{definition}
\newtheorem{definition-thm}[theorem-thm]{Definicja}
\newtheorem{corollary-thm}[theorem-thm]{Wniosek}
\newtheorem{remark-thm}[theorem-thm]{Obserwacja}
\newtheorem{problem-thm}{Problem}[section]

\newenvironment{theorem}[2]
  {\ifthenelse{\not\equal{#1}{}}{\begin{theorem-thm}[{\citealt[#2]{#1}}]}{\begin{theorem-thm}}}
  {\end{theorem-thm}}

\newenvironment{lemma}[2]
  {\ifthenelse{\not\equal{#1}{}}{\begin{lemma-thm}[{\citealt[#2]{#1}}]}{\begin{lemma-thm}}}
  {\end{lemma-thm}}

\newenvironment{definition}[2]
  {\ifthenelse{\not\equal{#1}{}}{\begin{definition-thm}[{\citealt[#2]{#1}}]}{\begin{definition-thm}}}
  {\end{definition-thm}}

\newenvironment{corollary}[2]
  {\ifthenelse{\not\equal{#1}{}}{\begin{corollary-thm}[{\citealt[#2]{#1}}]}{\begin{corollary-thm}}}
  {\end{corollary-thm}}

\newboolean{show-problem-refs}
\setboolean{show-problem-refs}{false}
\newenvironment{problem}[2]
  {\ifthenelse{\boolean{show-problem-refs} \and \not\equal{#1}{} }{\begin{problem-thm}[{\citealt[#2]{#1}}]}{\begin{problem-thm}}}
  {\end{problem-thm}}

\usepackage[cache=false,newfloat]{minted}
\newenvironment{code}{\captionsetup{type=listing}}{}
\SetupFloatingEnvironment{listing}{name=Algorytm}
\usepackage{caption}
\usepackage{csquotes}
\usepackage[titlenumbered,ruled]{algorithm2e}
\usepackage{algpseudocode}
\renewcommand*{\algorithmcfname}{Problem}
\SetKwInOut{Input}{Wejście}
\SetKwInOut{Output}{Wyjście}

\usepackage{tikz}
\usepackage{xcolor}
\usetikzlibrary{chains,fit,shapes,decorations.pathreplacing}
\tikzstyle{tmtape}=[draw,minimum size=0.6cm]
\definecolor{lightgrey}{RGB}{192,192,192}

% Various notations
\newcommand{\E}{\mathbb{E}}
\newcommand{\Var}{\mathrm{Var}}
\newcommand{\cN}{\mathcal{N}}
\newcommand{\N}{\mathbb{N}}
\renewcommand{\Re}{\mathbb{R}}
\newcommand{\rank}[1]{\ensuremath{\textrm{rank}({#1})}}
\DeclareMathOperator{\A}{\mathcal{A}}
\DeclareMathOperator{\B}{\mathcal{B}}

% Function names with hyphens
\mathchardef\mhyphen="2D
\newcommand{\getmin}{get\mhyphen min}
\newcommand{\removemin}{remove\mhyphen min}
\newcommand{\decreasekey}{decrease\mhyphen key}
\newcommand{\insertfront}{insert\mhyphen front}
