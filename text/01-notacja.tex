\section{Notacja}

\subsection{Podstawowe pojęcia}

\begin{definition}{}{}
  {\bf\textit{Alfabet}} $\A$ -- (skończony) zbiór symboli.
\end{definition}

W większości algorytmów rozmiar alfabetu jest stały i pomijany w analizie złożoności.

\begin{definition}{}{}
  {\bf\textit{Słowo}} $w \in \A^*$ to ciąg zbudowany nad alfabetem $\A$.
\end{definition}

Słowo puste jest oznaczane symbolem $\varepsilon$. Zbiór słów niepustych to $\A^+ = \A^* \setminus \{\varepsilon\}$.

Zbiór słów długości $n$ oznaczamy $\A_n$ i definiujemy jako $\A_0 = \{\varepsilon\}$, $A_{n + 1} = \Cup_{x \in \A_n, y \in \A} xy$. 

Numerację symboli w słowie zaczynamy od $1$, więc $w[i]$ (albo $w_i$) jest $i$-tym symbolem w słowie $w$.

\begin{definition}{}{}
  {\bf\textit{Długość słowa}} to funkcja $|\cdot|: \A^* \to \N$  taka, że $|w| = n$ wtedy i tylko wtedy, gdy $w \in \A^n$.
\end{definition}

\begin{definition}{}{}
  Słowo $u$ jest {\bf\textit{podsłowem}} (ang. \emph{factor}) słowa $w$, gdy istnieją słowa $v_1 v_2 \in \A^*$ takie, że $w = v_1 u v_2$.
  Podsłowo jest {\bf\textit{właściwe}}, gdy $v_1 v_2 \neq \varepsilon$.
\end{definition}

% Zbiór wszystkich podsłów oznaczamy $F(w) = \{u: \exists_{v_1, v_2: v_1 v_2\neq\varepsilon} v_1 u v_2\}$.
% Zbiór podsłów długości $k$ oznaczamy $F_k(w) = \{u: u \in F(w) \land |u| = k\}$.

\begin{definition}{}{}
  Słowo $u$ jest {\bf\textit{prefiksem}} ({\bf\textit{sufiksem}}) słowa $w$, gdy istnieje słowo $v \in \A^*$ takie, że $w = u v$ ($w = v u$).
  Prefiks (sufiks) jest {\bf\textit{właściwy}}, gdy $v \neq \varepsilon$.
\end{definition}

\begin{definition}{}{}
  Słowo $u$ jest {\bf\textit{podciągiem}} słowa $w$, gdy istnieją liczby $1 \le i_1 < i_2 < \ldots < i_k \le |w|$ takie, że $u = w[i_1] w[i_2] \ldots w[i_k]$.
\end{definition}

\begin{definition}{}{}
  Słowo $\bar{w}$ jest {\bf\textit{odwrotnością}} słowa $w$, gdy dla $n = |w|$ mamy $\bar{w} = w[n] w[n - 1] \ldots w[1]$.
\end{definition}

\begin{definition}{}{}
  Słowo $w$ jest {\bf\textit{palindromem}}, jeśli istnieją słowa $u, v \in \A^*$ takie, że $|u| \le 1$ i $w = \bar{v} u v$.
\end{definition}

\subsection{Porządki i odległości}

Złożoność obliczeniowa algorytmów tekstowych obliczana jest przy przyjęciu dwóch (binarnych) operacji atomowych na parach liter z alfabetu: $=$ oraz $\le$.

\begin{definition}{}{}
  Relacja $\preceq$ jest relacją {\bf\textit{porządku prefiksowego}} tj. $u \preceq v$ wtedy i tylko wtedy, gdy $u$ jest prefiksem $u$.
\end{definition}

\begin{definition}{}{}
  Relacja $\preceq_R$ jest relacją {\bf\textit{porządku wojskowego}} (\emph{radix}) tj. $u \preceq_R v$ wtedy i tylko wtedy, gdy:
  \begin{itemize}
    \item albo $|u| < |v|$,
    \item albo $|u| = |v|$ oraz istnieje $1 \le k \le |u|$ takie, że $u[k] < v[k]$ i dla wszystkich $1 \le j < k$ zachodzi $u[j] = v[j]$.
  \end{itemize}
\end{definition}

\begin{definition}{}{}
  Relacja $<$ jest relacją {\bf\textit{porządku leksykograficznego}} tj. $u < v$ wtedy i tylko wtedy, gdy:
  \begin{itemize}
    \item albo $u$ jest prefiksem $v$,
    \item albo istnieje $1 \le k \le |u|$ takie, że $u[k] < v[k]$ i dla wszystkich $1 \le j < k$ zachodzi $u[j] = v[j]$.
  \end{itemize}
\end{definition}

\begin{corollary}{}{}
  Porządek leksykograficzny i wojskowy są porządkami liniowymi, rozszerzającymi porządek prefiksowy. Dla słów równego długości $u$, $v$ zachodzi $u < v$ wtedy i tylko wtedy, gdy $u \preceq_R v$.
\end{corollary}

\subsection{Okresy i słowa pierwotne}

\begin{definition}{}{}
  Liczba $p \in \N_+$ jest {\bf\textit{okresem}} słowa $w$, jeśli dla każdego $1 \le i \le |w| - p$ zachodzi $w[i] = w[i + p]$.
  \\
  Najmniejszy okres słowa $w$ oznaczamy przez $p(w)$. Z definicji $|w|$ jest okresem słowa, więc $p(w)$ jest dobrze zdefiniowane.
\end{definition}

% Zbiór wszystkich okresów słowa $w$ oznaczamy przez $\Pi(w)$.

\begin{definition}{}{}
  Słowo $u$ jest {\bf\textit{prefikso-sufiksem}} (ang. \emph{border}) słowa $w$, jeśli istnieją słowa $v_1, v_2$ takie, że $w = u v_1 = v_2 u$.
  \\
  Z definicji $\varepsilon$ jest prefikso-sufiksem każdego słowa $w$.
\end{definition}

\begin{problem}{crochemore2002jewels}{s. 12}
  Pokaż, że następujące warunki są równoważne:
  \begin{enumerate}[label=(\roman*)]
    \item $w$ ma okres $p$,
    \item $w$ jest podsłowem pewnego $v^k$ dla $|v| = p$ i $k \ge 1$,
    \item $w = (uv)^kv$ dla $|uv| = p$, $v \neq \varepsilon$ i $k \ge 1$,
    \item $w$ ma prefikso-sufiks długości $|w| - p$.
  \end{enumerate} 
\end{problem}

\begin{definition}{}{}
  Słowo $w$ jest {\bf\textit{pierwotne}}, jeśli nie istnieje słowo $v$ oraz liczba całkowita $k \ge 2$ takie, że $w = v^k$.
\end{definition}

\begin{problem}{lothaire2002algebraic}{Problem 1.2.1, s. 40}
  Słowo $w$ jest słowem pierwotnym wtedy i tylko wtedy, gdy $p(w) = |w|$ lub $p(w)$ nie dzieli $|w|$.
\end{problem}

\begin{problem}{lothaire2002algebraic}{Problem 8.1.6}
  Pokaż, że następujące twierdzenia są równoważne:
  \begin{enumerate}[label=(\roman*)]
    \item $p(w^2) = |w|$,
    \item $w$ jest słowem pierwotnym,
    \item $w^2$ zawiera dokładnie dwa wystąpienia $w$.
  \end{enumerate} 
\end{problem}

\begin{theorem-thm}[Słaby lemat o okresowości]
  Jeśli słowo $w$ ma okresy $p$ i $q$ takie, że $|w| \ge p + q$, to $w$ ma również okres $NWD(p, q)$.
\end{theorem-thm}

\begin{proof}
  Bez straty ogólności załóżmy, że $p \ge q$.
  Najpierw wykażemy, że jeżeli słowo $w$ ma okresy $p$ i $q$ oraz $|w| \ge p + q$, to $w$ ma również okres $p - q$.
  
  Dla $1 \le i \le q$ mamy $a[i] = a[i + p] = a[i + p - q]$, ponieważ $1 \le i \le i + p - q \ge i + p \le |w|$.
  Podobnie dla $q + 1 \le i \le |w| - (p - q)$ mamy $a[i] = a[i - q] = a[i + p - q]$, ponieważ $1 \le i - q \le i + p - q \ge |w|$.

  Z algorytmu Euklidesa wynika wprost, że iterując to rozumowanie możemy pokazać, że $w$ ma również okres $NWD(p, q)$.
\end{proof}

\begin{theorem-thm}[Silny lemat o okresowości]
  Jeśli słowo $w$ ma okresy $p$ i $q$ takie, że $|w| \ge p + q - NWD(p, q)$, to $w$ ma również okres $NWD(p, q)$.
\end{theorem-thm}

\begin{proof}%[\citealx{Theorem 8.1.4, s. 272}{lothaire2002algebraic}}]
  Po pierwsze, jeśli słowo $w$ ma okresy $0 < q < p \le |w|$, to prefiks i sufiks $w$ o długości $|w| - q$ mają okresy $p - q$.
  Dla prefiksu wystarczy zauważyć, że dla dowolnego $1 \le i \le |w| - p$ zachodzi $w[i] = w[i + p] = w[i + p - q]$ -- a to właśnie jest definicja okresu dla słowa $w[1..(|w| - q)]$. 

  Po drugie, jeśli $w$ ma okres $q$ i istnieje podsłowo $v$ słowa $w$ z $|v| \ge q$ takim, że $r$ jest okresem $v$ i $r$ dzieli $q$, to $w$ ma okres $r$.
  
  Niech $r = NWD(p, q)$. Dowód przebiega przez indukcję ze względu na $s = \frac{p + q}{r}$. Dla $p = q = r$ -- więc twierdzenie jest oczywiście spełnione np. dla $s = 2$.
  
  Dla $s > 2$ i $q < p$ weźmy słowo $w$ mające okresy $p$ i $q$ takie, że $|w| \ge p + q - r$. Niech $u = w[1..q]$ oraz $w = uv$. Wówczas z pierwszego faktu wiemy, że $v$ ma okres $p - q$.
  Jednocześnie $v$ jest podsłowem $w$ oraz $|v| = |w| - q \ge p - r \ge q$, więc $v$ ma również okres $q$.
  
  Dalej wiemy, że $r = NWD(p - q, q)$, $s > \frac{(p - q) + q}{r}$ oraz
  \begin{align*}
    |v| = |w| - q \ge (p + q - r) - q = (p - q) + q - NWD(p - q, q).
  \end{align*}
  Z założenia indukcyjnego $v$ ma zatem okres $r$. Ostatecznie z drugiego faktu wiemy, że $w$ ma też okres $r$.
\end{proof}

\begin{problem}{lothaire2002algebraic}{Remark 8.1.5, s. 272}
  Korzystając ze słów Fibonacciego pokaż, że warunku $|w| \ge p + q - NWD(p, q)$ w silnym lemacie o okresowości nie da się poprawić.
\end{problem}

\begin{definition}{}{}
  Liczba $ord(w)$ jest {\bf\textit{rzędem}} słowa $w$, jeśli $ord(w) = |w|/p(w)$.
\end{definition}

\begin{problem}{}{}
  Pokaż, że słowo Fibonacciego jest słowem pierwotnym.
\end{problem}

\begin{proof}
Załóżmy, że istnieją $u, k: u^k = Fib_i, k > 1$. Wprowadźmy oznaczenie $u = st$ gdzie $s = u[1\ldots|u|-2]$ oraz $t = u[|u|-1,|u|]$. 

Na podstawie poprzedniego zadania wiemy, że słowa Fibonacciego bez ostatnich dwu znaków są palindromem. Dlatego $(st)^{k-1}s$ powinno też być palindromem. Z~tego wynika, że $s = \overline{s}$ oraz $t = \overline{t}$. To może zachodzić tylko dla dwu przypadków: $t = 00$ i $t = 11$. Na ćwiczeniach jednak zostało pokazane, że dla słów Fibonacciego jedyne dwie opcje są $t = 01$ i $t = 10$. Otrzymujemy sprzeczność, która pokazuje, że żadne słowo Fibonacciego nie można zapisać jako $u^k, k > 1$, więc każde słowo Fibonacciego jest słowem pierwotnym.
\end{proof}

\subsection{Słowa sprzężone}

\begin{definition}{}{}
  Słowa $v, w$ są {\bf\textit{sprzężone}} wtedy i tylko wtedy, gdy istnieją słowa $u_1$, $u_2$ takie, że $v = u_1 u_2$ i $w = u_2 u_1$.
\end{definition}

\begin{corollary}{}{}
  Relacja sprzężenia (cyklicznego obrotu) jest relacją równoważności, więc definiuje klasy równoważności w $\A^*$.
\end{corollary}

\begin{problem}{lothaire2002algebraic}{}
  Pokaż, ile elementów ma klasa równoważności dla słowa $v$.
\end{problem}

\begin{problem}{crochemore2002jewels}{s. 17}
  Pokaż dowód małego twierdzenia Fermata na bazie wiedzy, że słowo pierwotne $v$ należy do klasy równoważności o mocy $|v|$.
\end{problem}

\begin{proof}
Przypomnijmy na wstępie dowodu wypowiedź małego twierdzenia Fermata: jeżeli $p$ jest liczbą pierwszą, a $n$ dowolną liczbą naturalną, to $p | (n^p-n)$.

Zdefiniujmy taką relację na słowach, że $x$ jest w relacji z $y$, jeżeli tylko $x$ jest cyklicznym przesunięciem $y$. Oczywiście jest to relacja równoważności. Rozważmy słowa unarne długości $p$, czyli słowa postaci $a^p$, gdzie $a$ jest pewną literą. Niech $K$ będzie zbiorem wszystkich nieunarnych słów długości $p$ nad alfabetem $\{1,\cdots,n\}$. Wszystkie te słowa są pierwotne, ponieważ ich długość jest liczbą pierwszą oraz nie są unarne. Na podstawie założeń dostajemy, że każda klasa równoważności naszej relacji ma dokładnie $p$ elementów. Dodatkowo zauważmy, że zbiór $K$ ma dokładnie $n^p-n$ elementów. Ponieważ $K$ może być podzielony na rozłączne podzbiory mające po $p$ elementów, to $p|(n^p-n)$, co kończy dowód.
\end{proof}

\begin{definition}{}{}
  Słowo $w$ jest {\bf\textit{słowem Lyndona}} wtedy i tylko wtedy, gdy jest minimalnym (leksykograficznie, wojskowo) słowem w ramach klasy sprzężenia.
\end{definition}

\subsection{Morfizmy i klasy słów}

\begin{definition}{}{}
  Funkcja $f: \A^* \to \B^*$ jest {\bf\textit{morfizmem}} ({\bf\textit{podstawieniem}}) jeśli $f(xy) = f(x)f(y)$ dla wszystkich $x, y \in \A^*$.
  \\
  Morfizm jest dosłowny (\emph{literal}), gdy dla każdego $x \in \A$ zachodzi $|f(x)| = 1$.
  \\
  Morfizm jest nieusuwający (\emph{non-erasing}), gdy dla każdego $x \in \A$ zachodzi $f(x) \neq \varepsilon$.
\end{definition}

\subsubsection{Słowa Fibonacciego}

\begin{definition}{lothaire2002algebraic}{s. 10-11}
  Słowo $f_n$ jest {\bf\textit{słowem Fibonacciego}}, gdy $f_0 = 0$, $f_1 = 01$ oraz $f_{k + 2} = f_{k + 1} f_k$ dla $k = 0, 1, \ldots$.
  \\
  Równoważnie, $f_k = \phi^n(0)$ dla morfizmu $\phi(0) = 01$, $\phi(1) = 0$.
\end{definition}

\begin{problem}{}{}
  Dla $k \ge 1$ niech $u = f_k f_{k + 1}$ i $v = f_{k + 1} f_k$. Pokaż, że $u$ powstaje z $v$ przez zamianę dwóch ostatnich liter.
\end{problem}

\begin{problem}{lothaire2002algebraic}{Problem 8.2.7, s. 308}
  Jeśli dla $u \in \A^+$ słowo $u^2$ jest podsłowem pewnego $f_k$, to $|u|$ jest pewną liczbą Fibonacciego i $u$ jest sprzężone z pewnym $f_l$.
\end{problem}

\subsubsection{Słowa Thuego-Morse'a}

\begin{definition}{lothaire2002algebraic}{s. 11}
  Słowo $u_n$ jest {\bf\textit{słowem Thuego-Morse'a}}, gdy $u_0 = 0$, $v_0 = 1$ oraz $u_{k + 1} = u_k v_k$, $v_{k + 1} = v_k u_k$ dla $k = 0, 1, \ldots$.
  \\
  Równoważnie, $u_k = \mu^n(0)$ dla morfizmu $\mu(0) = 01$, $\mu(1) = 10$.
\end{definition}

\begin{problem}{}{}
  Udowodnić że słowa Thuego-Morse'a $u_{n}$ są słowami pierwotnymi.
\end{problem}

\begin{problem}{}{}
  Udowodnić że słowa Thuego-Morse'a $u_{2n}$ są palindromami.
\end{problem}

\begin{proof}
Zgodnie z definicją rekurencyjną mamy $u_0=0,v_0=1$ oraz $u_{k+1}=u_kv_k$, $v_{k+1}=v_ku_k$. Rozumować będziemy indukcyjnie. Dla $n=0$, słowo $u_0=0, v_0=1$ są oczywiście palindromami. Załóżmy zatem, że dla wszystkich $k < n$ prawdą jest, że $u_{2k}$ oraz $v_{2k}$ są palindromami. Będziemy chcieli wykazać, że jest to prawdą także dla $u_{2n}$ i $v_{2n}$. Z definicji rekurencyjnej dostajemy, że $u_{2n}=u_{2n-1}v_{2n-1}=u_{2n-2}v_{2n-2}v_{2n-2}u_{2n-2}$. Ponieważ $u_{2n_2}$ oraz $v_{2n-2}$ są palindromami na mocy założenia indukcyjnego, to $u_{2n}$ też jest palindromem. Dokładnie taki sam argument pokazuje, że $v_{2n}$ również jest palindromem. Zatem zakończyliśmy dowód kroku indukcyjnego, a więc też cały dowód.
\end{proof}

\begin{problem}{lothaire2002algebraic}{3.1.1, s. 113-114}
  Sprawdzić czy słowa Thuego-Morse'a zawierają podsłowa $u^3$ lub $(uv)^2u$ dla pewnych $u, v \in \A^+$.
\end{problem}

\subsection{Kody}

\begin{definition}{}{}
  $X \subset \A^+$ jest {\bf\textit{kodem}}, gdy dla dowolnych $x_1, \ldots, x_n, y_1, \ldots, y_m \in X$ jeśli dla $x_1 \ldots x_n = y_1 \ldots y_m$, to $n = m$ oraz $x_i = y_i$ dla $i = 1, \ldots, n$.
\end{definition}

\begin{definition}{}{}
  $X \subset \A^+$ jest {\bf\textit{kodem prefiksowym}} gdy $X$ jest kodem i dla żadnych $x, y \in X$ słowo $x$ nie jest prefiksem $y$.
\end{definition}

\begin{problem}{lothaire2002algebraic}{6.1.3, s. 198-199}
  Zbiór $X \subset \A^+$ jest kodem dla $\B^*$ wtedy i tylko wtedy, gdy dowolny morfizm $\phi: \B^* \to \A^*$ indukuje iniekcję z $\B$ na $X$.
\end{problem}

% TODO: Lothaire, Algebraic: 6.2.10, 6.2.11 [s. 226] + generalnie cały rozdział 6