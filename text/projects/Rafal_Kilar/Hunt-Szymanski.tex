\documentclass[11pt]{article}
\usepackage[letterpaper,top=2cm,bottom=2cm,left=3cm,right=3cm,marginparwidth=1.75cm]{geometry}
\usepackage{amsmath}
\usepackage{graphicx}
\usepackage[colorlinks=true, allcolors=blue]{hyperref}
\usepackage{algorithm}
\usepackage{algpseudocode}
\usepackage[polish]{babel}
\usepackage[utf8]{inputenc}
\usepackage{polski}
\usepackage[lf,enc=t1]{berenis}

\setlength{\parskip}{0.5em}
\renewcommand{\baselinestretch}{1}

\title{Projekt Algorytmy Tekstowe}
\author{Rafał Kilar}

\begin{document}

\maketitle

\section*{Wstęp}

Rozważamy problem najdłuższego wspólnego podciągu. Mamy dane dwa słowa $A[1..m]$ i $B[1..n]$. Przedstawimy algorytm Hunta-Szymanskiego działający w czasie $O((r+n) \log{n} + n)$ dla słów z $r$ dopasowaniami i zakładając $m = n$. Następnie przedstawimy jego usprawnioną wersję działającą w czasie $O(m\log{n} + d \log{(2mn/d)} + n)$ dla słów z $d$ dominującymi dopasowaniami.

\section*{Algorytm Hunta-Szymanskiego}

Algorytm i omówienie na podstawie pracy \cite{hunt1977fast}.

Definiujemy wartości progowe $T_{i,k}$ jako 
$$T_{i,k} =  min\{j\colon \text{$A[i..i]$ i $B[1..j]$ posiadają wspólny podciąg długości $k$}\}$$
Możemy udowodnić kilka prostych faktów o wartościach progowych:

\textbf{Lemat 1.} \textit{Jeśli $T_{i,1}, \dots, T_{i, k}$ są zdefiniowane, to $T_{i,1} < \dots < T_{i,k}$.}

\textit{Dowód.} Zauważmy, że $B[T_{i,k}]$ jest ostatnim znakiem pewnego wspólnego podciągu $A[1..i]$ i $B[1..T_{i,k}]$. W przeciwnym przypadku $T_{i,k}$ nie byłoby minimalne. Wiemy więc, że $A[1..i]$ i $B[1..T_{i,k}-1]$ mają wspólny podciąg długości $k-1$, zatem $T_{i,k-1} \le T_{i,k}-1$.

Lemat ten przyda nam się w konstrukcji algorytmu.

\textbf{Lemat 2.} $T_{i,k-1} < T_{i+1, k} \le T_{i,k}$

\textit{Dowód.} Ponieważ $A[1..i]$ i $B[1..T_{i,k}]$ mają wspólny podciąg długości $k$, to mają je także $A[1..i+1]$ i $B[1..T_{i,k}]$, więc $T_{i+1, k} \le T_{i,k}$. $A[1..i+1]$ i $B[1..T_{i+1,k}]$ mają wspólny podciąg długości $k$. Usunięcie końcowych znaków z tych słów zmniejsza długość najdłuższego wspólnego podciągu o co najwyżej $1$, więc $A[1..i]$ i $B[1..T_{i+1,k}-1]$ mają wspólny podciąg długości $k-1$, czyli $T_{i,k-1} \le T_{i+1,k} - 1$.

Używając tego lematu możemy wyznaczyć wartości $T_{i,k}$.

\textbf{Lemat 3.}

$$T_{i+1,k} = \begin{cases}
    \min\{j\colon A[i+1] = B[j] \land T_{i, k-1} < j \le T_{i,k}\} & \text{jeśli takie $j$ istnieje} \\
    T_{i, k} & \text{wpp.} 
\end{cases}$$

\textit{Dowód.} Rozważamy przypadek, że takie $j$ nie istnieje. Z minimalności $T_{i+1,k}$, każdy maksymalny wspólny podciąg $A[1..i+1]$ i $B[1..T_{i+1,k}]$ kończy się w $B[T_{i+1,k}]$. Z założenia, że szukane $j$ nie istnieje, wiemy że $A[i + 1] \neq B[j]$ dla każdego $T_{i, k-1} < j \le T_{i,k}$. Z lematu 2. mamy natomiast $T_{i+1,k} \le T_{i,k}$, więc $B[T_{i+1,k}] \neq A[i+1]$, czyli rozważany podciąg jest także podciągiem $A[1..i]$ i $B[1..T_{i+1,k}]$, więc $T_{i,k} \le T_{i+1,k}$ i z lematu 2. $T_{i,k} = T_{i+1,k}$.

Załóżmy teraz, że takie minimalne $j$ istnieje. $A[1..i+1]$ i $B[1..j]$ mają wspólny podciąg długości $k$ - wspólne podciąg długości $k-1$ $A[1..i]$ i $B[1..T_{i,k-1}]$ powiększone o $A[i+1]$ i $B[j]$, więc $T_{i+1,k} \le j$. 

Załóżmy niewprost, że $T_{i+1,k} < j$. Z lematu 2. mamy $T_{i,k-1} < T_{i+1,k}$. Z definicji $j$ zachodzi $A[i+1] \neq B[l]$ dla każdego $T_{i,k-1} < k < j$, w szczególności $B[T_{i+1,k}] \neq A[i+1]$. Otrzymujemy więc, że ostatni znak najdłuższego wspólnego podciągu $A[1..i+1]$ i $B[1..T_{i+1,k}]$ długości $k$ nie jest równy $A[i+1]$. $A[1..i]$ i $B[1..T_{i+1,k}]$ muszą również mieć podciąg długości $k$, więc $T_{i,k} \le T_{i+1,k}$. Z lematu 2. mamy więc równość $T_{i, k} = T_{i+1,k}$. Z założenia niewprost i wymogu $T_{i+1,k} < j \le T_{i,k}$ mamy $T_{i,k} \neq T_{i+1,k}$, co daje nam sprzeczność i kończy dowód.

\begin{algorithm}
\caption{Algorytm HS}\label{alg:hs}
\begin{algorithmic}
\For{$i = 1 .. m$}
    \State $MATCHLIST[i] = \{j_1, j_2, \dots, j_p\} \text{ such that $j_1 > \dots > j_p$ and $A[i] = B[j_k]$}$
    \State $THRESH[i] = m + 1$
\EndFor
\State $THRESH[0] = 0$
\State $LINK[0] = 0$
\For{$i = 1 .. m$}
    \For{$j \text{ in } MATCHLIST[i]$}
        \State $\text{find $k$ such that} THRESH[k-1] < j < THRESH[k]$
        \If{$j < THRESH[k]$}
            \State $THRESH[k] = j$
            \State $LINK[k] = newnode(i, j, LINK[k-1])$
        \EndIf
    \EndFor
\EndFor
\State recover LCS in reverse using the $LINK$ array
\end{algorithmic}
\end{algorithm}

W naszym algorytmie będziemy wyznaczać kolejne wiersze $T_{i,1}, \dots$.

Algorytm będzie iterował się po kolejnych znakach $A$. Na początku każdej iteracji tablica $THRESH$ będzie zawierała kolejne wartości $T_{i,1}, T_{i,2}, \dots$. Lista $MATCHLIST[i]$ zawiera wszystkie indeksy $j$ takie, że $B[j] = A[i]$ w malejącej kolejności. 

Załóżmy, że dla w iteracji $i$ algorytm rozważa dopasowania $A[i]$ do $B[j_1], \dots, B[j_p]$ pomiędzy wartościami progowymi, to jest $THRESH[k-1] = T_{i-1,k-1} < j_1 < \dots < j_p \le T_{i-1, k} = THRESH[k]$. Z lematu 3. $T_{i,k} = j_1$. Ponieważ rozważamy indeksy w malejącej kolejności, poprawimy wartość $THRESH[k]$ na kolejne $j_p, j_{p-1}, \dots$ aż skończymy na $j_1$.

Dla prostoty zapisu załóżmy $n=m$. Ponieważ lemat 1 mówi nam, że wartości w $THRESH$ są w kolejności rosnącej, możemy wyszukiwać $k$ używając wyszukiwania binarnego w czasie $O(\log{n})$. Jeśli oznaczymy ilość wszystkich dopasowań pomiędzy $A$ i $B$ przez $r$, to wykonanie głównej pętli zajmuje $O(r\log{n} + n)$. Wyznaczenie list $MATCHLIST$ możemy wykonać na przykład przez posortowanie par znaków i indeksów z $A$ i $B$ i równoległe przejście po tak posortowanych tablicach - wyznaczamy listy indeksów dla rosnących znaków z $B$ i przypisujemy je do każdego odpowiadającemu im indeksowi z $A$. Taka implementacja działa w czasie $O(n\log{n})$. Potrzebujemy $O(n)$ pamięci na wykorzystywane listy i tablice oraz $O(r)$ pamięci na linki. 

Łączna złożoność czasowa to $O((r+n)\log{n})$ i pamięciowa $O(r + n)$.

\clearpage

\section*{Usprawniony algorytm}

Przedstawimy teraz usprawnioną wersję powyższego algorytmu zaproponowaną w \cite{apostolico1986improving}. Zanim to zrobimy wprowadzimy kilka pojęć. Powiemy, że dopasowanie $[i,j]$ (para taka, że $A[i] = B[j]$) ma rząd $k$ jeśli $A[1..i]$ i $B[1..j]$ mają najdłuższy wspólny podciąg długości $k$. Dopasowanie $[i, j]$ nazwiemy $k$-dominującym jeśli $[i, j]$ ma rząd $k$ i dla każdego dopasowania $[i', j']$ o rzędzie $k$ zachodzi albo $i' > i \land j' \le j$ albo $i' \le i \land j' > j$. Liczbę wszystkich dominujących dopasowań oznaczymy przez $d$. Na poniższym rysunku dominujące dopasowania oznaczone są czerwonymi okręgami.

\begin{center}
    \includegraphics[width=0.45\textwidth]{Lmat.png}
\end{center}

Aby przyśpieszyć algorytm \ref{alg:hs} zmodyfikujemy proces wyznaczania $k$-dominujących dopasowań. Nie będziemy sprawdzać wszystkich dopasowań. Zamiast tego, przechowujemy dla każdego symbolu $\sigma$ listę $MATCHLIST[\sigma]$ listę aktywnych indeksów, to jest takich, które nie wyznaczają wartości progowych. Będziemy wyznaczać jedynie dominujące dopasowania. W tym celu posłużą nam operacje na słownikach:

\begin{itemize}
    \item $SEARCH(key, LIST)$ zwraca najmniejszy element listy $LIST$ nie mniejszy niż $key$ lub $n+1$ jeśli taki element nie istnieje. $SEARCH(n+1, LIST)$ od razu zwraca $n+1$.
    \item $INSERT(key, LIST)$ i $DELETE(key, LIST)$ odpowiednio dodają i usuwają wartość $key$ z listy $LIST$. Jeśli $key = n+1$, to nic nie robią.
    \item $first(LIST)$ zwraca najmniejszy element w liście $LIST$
\end{itemize}

Algorytm przedstawiono poniżej (\ref{alg:hs1}). W zewnętrznej pętli iterujemy się po kolejnych znakach $A$. W każdej iteracji wewnętrznej pętli algorytm rozpatruje kolejne dominujące dopasowania $[i, j]$, wyszukuje próg $T$, który ma zastąpić $j$. Następnie aktualizuje listę $THRESH$ i listy aktywnych dopasowań $AMATCHLIST[\sigma]$ i zapamiętuje linki. Następnie wyszukuje kolejne dominujące dopasowanie $[i, j]$ dla $j >= T$ w liście aktywnych dopasowań $AMATCHLIST[A[i]]$.

W wewnętrznej pętli zachowujemy następujący niezmiennik: jeśli $j \ne n + 1$, wtedy $[i, j]$ jest $k$-dominującym dopasowaniem i pierwsze $k-1$ pozycji w $THRESH$ zawiera poprawne wartości dla $i$-tego wiersza. Po $i$-tej iteracji zewnętrznej pętli następujące niezmienniki są zachowane:

\begin{itemize}
    \item $THRESH$ zawiera wartości $T_{i, 1}, \dots, T_{i,l}$
    \item $AMATCHLIST[\sigma]$ zawiera indeksy $\sigma$ w $B$, które nie są w $THRESH$
\end{itemize}

\begin{algorithm}
\caption{Algorytm HS1}\label{alg:hs1}
\begin{algorithmic}
\For{$i = 1 .. m$}
    \State $\sigma = A[i]$
    \State $j = first(AMATCHLIST[\sigma])$
    \State $FLAG = true$
    \While{$FLAG$}
        \State $T = SEARCH(j, THRESH)$
        \State $k = rank(T)$
        \If{$T = n + 1$}
            \State $FLAG = false$
        \EndIf
        \State $INSERT(j, THRESH)$
        \State $DELETE(T, THRESH)$
        \State $LINK[k]=  newnode(i, j, LINK[k-1])$
        \State $\sigma' = B[T]$
        \State $DELETE(j, AMATCHLIST[\sigma])$
        \State $j = SEARCH(T, AMATCHLIST[\sigma])$
        \State $INSERT(T, AMATCHLIST[\sigma'])$
    \EndWhile
\EndFor
\State recover LCS in reverse using the $LINK$ array
\end{algorithmic}
\end{algorithm}

Czas działania zależy od implementacji list $THRESH$ i $MATCHLIST[\sigma]$. Skorzystamy ze struktury danych nazwanej C-drzewo opisanej w pracy \cite{apostolico1987longest}. Reprezentuje ona uporządkowaną listę elementów z ustalonego uniwersum $U$ jako statyczne drzewo binarne. Kolejne liście odpowiadają kolejnym wartościom z $U$. Dodatkowo utrzymujemy wskaźniki na niedawno odwiedzone liście. Pozwala to na szybsze wyszukiwanie kolejnych wartości - możemy zacząć od zapamiętanego liścia i przejść najpierw w górę i później w dół po najkrótszej ścieżce do szukanego liścia. Po zakończeniu operacji przestawiamy ten wskaźnik.

Możemy pokazać, że jeśli wykonujemy operacje na liściach $i_0 < i_1 < \dots < i_k$, gdzie $b_j = i_j - i_{j-1}$ operacje te wymagają czasu $O(\log{m} + \sum_{j=1}^k \log{b_j})$. Pierwsza operacja działa w czasie $O(\log{m})$, gdyż wskaźnik może wskazywać na późniejszy liść. Musimy wtedy przestawić wskaźnik na pierwszy liść. Następne operacje przechodzą po najkrótszej ścieżce między liśćmi, mają one długość $O(\log{b_j})$. 

W $i$-tej iteracji zewnętrznej pętli znajdujemy $d_i$ dominujących dopasowań. Operacja na określonej liście zajmują $O(\log{n} + \sum_{j=1}^{d_i} \log{b_j})$ czasu. Ponieważ rozmiar list to co najwyżej $n$, to $\sum_{j=1}^{d_i} b_j \le 2n$. We wszystkich iteracjach wykorzystywany czas wynosi $O(m\log{n} + \sum_{j=1}^d b_j)$, gdzie $\sum_{j=1}^d b_j \le 2nm$. Czas jest maksymalizowany gdy $b_j = 2nm/d$. 

Czas działania algorytmu wynosi więc $O(m\log{n} + d\log{(2nm/d)} + n)$ gdy uwzględnimy liniowy czas potrzebny na inicjalizację. 

\bibliographystyle{alpha}
\bibliography{sample}

\end{document}