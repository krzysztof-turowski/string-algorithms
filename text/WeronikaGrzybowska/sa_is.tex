\documentclass[12pt]{article}
\usepackage[polish]{babel}
\usepackage[utf8]{inputenc}
\usepackage[T1]{fontenc}
\usepackage{amsthm}

\newtheorem{definition}{Definicja}
\newtheorem{lemma}{Lemat}
\newcommand{\elps}{\makebox[.8em][c]{.\hfil.\hfil.}}
\newcommand{\range}[2]{\{#1, \elps, #2\}}

\begin{document}

\section*{Algorytm SA-IS}

\subsection*{Notacja}

Jako $S$ będziemy oznaczać słowo, którego tablicę sufiksową mamy obliczyć.
Przyjmujemy, że $S$ kończy się znakiem \(\$\), który nie występuje nigdzie
indziej w słowie i w porządku leksykograficznym jest najmniejszy.

Dodatkowo niech $n$ będzie długością słowa $S$, a znaki będą numerowane od 1.
Wtedy $S_{i} = S[i..n]$ dla $i \in \range{1}{n}$ jest sufiksem $S$
rozpoczynającym się na $i$-tej pozycji.

\subsection*{Algorytm}

Ogólny sposób działania SA-IS polega na sortowaniu sufiksów na podstawie
tablicy sufiksowej krótszego słowa $S'$, obliczanej rekurencyjnie.
Słowo $S'$ powstaje przez wybór pewnych podsłów $S$, i przepisaniu każdego
z nich na pojedynczy znak zachowując kolejność leksykograficzną.
Aby osiągnąć czas $O(n)$, obie te czynności wykonywane są za pomocą
\textit{sortowania indukowanego}.

Aby przedstawić dokładne działanie algorytmu, koniecznie są następujące pojęcia:

\begin{definition}
	$S_i$ jest sufiksem typu S (typu L) gdy $S_i < S_{i+1}$ ($S_i > S_{i+1}$).

	\(S_n = \$\) jest typu S.

	Typ znaku $S[i]$ określany jest jako ten sam co typ sufiksu $S_i$.
\end{definition}

\begin{definition}
	Znak $S[i]$, $i \in \range{2}{n}$, jest LMS-znakiem (leftmost S-type) gdy
	$S[i-1]$ i $S[i]$ są odpowiednio typu L i typu S.
\end{definition}

\begin{definition}
	LMS-podsłowo to 1. \(S[n] = \$\); lub 2.\ podsłowo $S[i..j]$, gdzie
	$S[i]$, $S[j]$ to kolejne LMS-znaki w $S$.
\end{definition}

Podsłowa początkowo wybierane z $S$ to właśnie LMS-podsłowa. Sortowanie indukowane
zarówno dla LMS-podsłów (celem zredukowania podsłów do $S'$), jak i sufiksów $S$,
działa opierając się na podziale sufiksów na typ S i typ L.

Całość algorytmu zawiera się w poniższych krokach:

\begin{enumerate}
	\item Określ typ każdego z sufiksów $S$
	\item Wyznacz LMS-podsłowa $S$
	\item Posortuj znalezione LMS-podsłowa za pomocą sortowania indukowanego
	\item Przepisz każde LMS-podsłowo na literę odpowiadającą jego pozycji
		w porządku i utwórz z nich słowo $S'$ zachowując kolejność występowania
		w $S$
	\item Oblicz tablicę sufiksową $SA'$ słowa $S'$
	\item Wyznacz tablicę sufiksową $SA$ słowa $S$ za pomocą sortowania
		indukowanego na podstawie $SA'$
\end{enumerate}

W kroku 5, jeżeli nie występują dwa identyczne LMS-podsłowa (wszystkie znaki
$S'$ są unikalne), $SA'$ można uzyskać bezpośrednio jako odwrotność permutacji
reprezentowanej przez $S'$; w przeciwnym przypadku, jest ona pozyskiwana
rekurencyjnie.

\subsection*{Wyznaczanie LMS-podsłów}

Typ każdego z sufiksów można ustalić, przeglądając $S$ od końca:

\begin{enumerate}
	\item \$ jest typu S
	\item jeśli $S[i] = S[i+1]$, typ $S_i$ jest ten sam co $S_{i+1}$
	\item wpp. $S[i] \neq S[i+1]$, a typ $S_i$ zależy bezpośrednio od nich.
\end{enumerate}

Tablica przechowująca typy sufiksów $S$ pozwala wskazać pozycje, na których
zaczynają się kolejne LMS-podsłowa. W dalszej części algorytmu LMS-podsłowa
będą reprezentowane przez indeks ich pierwszego znaku.

\subsection*{Redukcja do mniejszego problemu}

Na LMS-podsłowach jest zdefiniowany następujący porządek: porównując parami
kolejne znaki obu podsłów, jeśli się różnią, decyduje ich kolejność
leksykograficzna; w przepiwnym przypadku znak typu L jest mniejszy od znaku
typu S. Ten porządek odpowiada temu, że jeśli $S_i$ jest typu S, $S_j$ typu L,
i pierwsze litery są sobie równe, to $S_j < S_i$.

Sortowanie indukowane pracuje na wynikowej tablicy $SA$, podzielonej
na kubełki dla każdego znaku w $S$. W trakcie sortowania wyłania się dodatkowy
podział każdego kubełka na te przenaczone dla sufiksów różnego typu w
kolejności L, S;\@ ale nie jest on reprezentowany bezpośrednio. Dla każdej
litery w tablicy $B$ rozmiaru alfabetu przechowywany jest wskaźnik na pewne
miejsce w odpowiadającym kubełku. Wskazywane miejsce zależy od etapu
sortowania.

Samo sortowanie odbywa się w trzech krokach:

\begin{enumerate}
	\item wyznacz koniec każdego z kubełków. Wstaw każde z LMS-podsłów do jego
		kubełka (odp.\ pierwszemu znakowi podsłowa);
	\item wyznacz początek każdego z kubełków. Dla każdego $i = 1, \elps, n$:
		jeśli $c = S[SA[i] - 1]$ jest typu L, wstaw $SA[i] - 1$ do kubełka $c$;
	\item wyznacz koniec każdego z kubełków. Dla każdego $i = n, \elps, 1$:
		jeśli $c = S[SA[i] - 1]$ jest typu S, wstaw $SA[i] - 1$ do kubełka $c$.
\end{enumerate}

Dokładniej, wyznaczenie początku (końca) kubełka $c$ oznacza zapisanie
odpowiedniego indeksu do $B[c]$, natomiast wstawianie elementu odbywa się przez
zapisanie go do $SA[B[c]]$ i przesunięcie $B[c]$ o jeden w lewo (prawo).

$S'$ jest utworzone przez ułożenie LMS-podsłów tak, jak występują w $S$ i
zamianę każdego z nich na kolejne liczby zgodnie z porządkiem uzyskanym z
powyższego sortowania (identyczne podsłowa otrzymują tę samą liczbę).

\subsection*{Sortowanie wszystkich sufiksów}

Procedura odtworzenia tablicy sufiksowej $S$ przez sortowanie indukowane jest
niemal identyczna z sortowaniem LMS-podsłów.  Jedyna różnica występuje w
pierwszym kroku: na końce kubełków należy wstawić LMS-podsłowa (indeksy
pierwszych znaków) w zgodzie z kolejnością uzyskaną w $SA'$. Po jego wykonaniu
zawartość $SA$ jest tablicą sufiksową $S$.

\subsection*{Poprawność sortowania}

Rozważmy ostatni krok całego algorytmu:

\begin{lemma}
	Mając dane posortowane wszystkie sufiksy typu L, krok 3 sortuje wszystkie
	sufiksy $S$ w czasie $O(n)$.
\end{lemma}

\begin{proof}
Pokażemy przez indukcję, że w momencie przeglądania $SA[i]$, $S_{SA[i]}$
został już zapisany na swoim miejscu.

\item Dla $i = n$, największy sufiks musi być typu L, zatem był posortowany w
kroku 2.

\item Dla $i < n$, gdy wszystkie $SA[i+1], \elps, SA[n]$ są uzupełnione
poprawnie, załóżmy że sufiks, który powinien znajdować się w $SA[i]$ leży
pod $SA[k]$, $k < i$. Ponieważ sufiksy typu L zostały posortowane, w
$SA[i]$, $SA[k]$ znajdują się sufiksy typu S. Sufiksy są już w swoich
kubełkach, więc $SA[i] = c\alpha$, $SA[k] = c\beta$ dla jakiegoś $c$.
$c\beta < \beta < \alpha$, zatem poprawne pozycje $\beta$, $\alpha$
znajdują się wśród już wstawionych i przeglądniętych $SA[i+1], \elps,
SA[n]$, a $c\alpha$ powinno było zostać wstawione na koniec swojego kubełka
przed $c\beta$, co jest sprzeczne z obecnym stanem tablicy $SA$.
\end{proof}

Gdyby najpierw posortować sufiksy typu S, a następnie wykonać krok 2, wszystkie
sufiksy typu L również zostały by posortowane, co można wykazać w analogiczny
sposób. Ze względu na to, że tylko LMS-sufiksy biorą udział w ich sortowaniu,
wystarczy zacząć od wstawienia do $SA$ LMS-sufiksów w kolejności
leksykograficznej, aby uzyskać ten sam rezultat. Tym samym, o ile tablica $SA'$
wyznacza poprawny porządek LMS-sufiksów, ostatni etap algorytmu jest poprawny.

Niech $P[i]$ wskazuje na początek $i$-tego LMS-podsłowa. Wtedy

\begin{lemma}
	$S'_i < S'_j$ jest równoważne z $S_{P[i]} < S_{P[j]}$.
\end{lemma}

\begin{proof}
	Jeśli $S'[i] \neq S'[j]$, to znak świadczący o różnicy w odpowiadających im
	LMS-podsłowach wyznacza kolejność sufiksów zaczynających się na tych samych
	pozycjach, natomiast porządek na LMS-podsłowach jest zgody z porzadkiem
	leksykograficznym na sufiksach. W przeciwynm wypadku, zauważmy, że $S'[i] =
	S'[j]$ oznacza równą długość odpowiednich LMS-podsłów, zatem
	możliwe jest przeprowadzenie powyższego rozumowania na pierwszym różniącym
	się znaku sufiksów w $S'$.
\end{proof}

Pozostaje wykazać, że pierwsze użycie sortowania indukowanego sortuje
LMS-podsłowa. Mechanizm dowodu jest podobny, z kilkoma różnicami: w pierwszym
kroku chcemy otrzymać poprawny porządek na LMS-podsłowach obciętych do
pierwszego znaku. Wystarczy, że zostaną wstawione do odpowiednich kubełków,
co zapewnia krok 1. Te z kolei wystarczą do posortowania sufiksów typu L
obciętych do pierwszeko LMS-znaku w kroku 2. Analogicznie dzieje się dla
sufiksów typu S w kroku 3, z wyjątkiem LMS-sufiksów, które są rozważane
do miejsca wystąpiena \textit{drugiego} LMS-znaku, czyli LMS-podsłów.

\subsection*{Złożoność}

Zarówno wyznaczanie LMS-podsłów, jak i sortowanie indukowane działa w oczywisty
sposób w czasie $O(n)$. Ponieważ LMS-podsłów jest nie więcej niż połowa
długości $S$ (z wyjątkiem \(\$\) w środku każdego z nich musi się znajdować
przynajmniej jeden znak typu L), całkowity czas wykonania można określić
równaniem $T(n) = T(\left \lfloor{n/2}\right \rfloor) + O(n)$, które daje
$O(n)$.

\end{document}
