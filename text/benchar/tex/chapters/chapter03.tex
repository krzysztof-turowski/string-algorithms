Implementacje algorytmów wykorzystane do pomiarów zaprezentowanych w poniższym rozdziale pochodzą z repozytorium: \\
\url{https://github.com/krzysztof-turowski/string-algorithms}. \\
W repozytorium tym (w folderze $benchar$ - pull request: \url{https://github.com/krzysztof-turowski/string-algorithms/pull/46}) znajduje się także kod źródłowy autorskiej biblioteczki służącej do pomiaru liczby porównań.

\section{Metodologia}
Zacznijmy od opisu sposobu pomiarów liczby porównań wykonywanej przez poszczególne algorytmy. W celu przeprowadzenia pomiarów stworzono klasę $benchar$. Poniżej przedstawiono związane z nią obiekty:
\begin{itemize}
    \item Obiekt klasy $benchar$:
    \begin{itemize}
        \item działa jako fabryka obiektów $count\_str$
        \item posiada pole $cmp\_count$, w którym znajduje się liczba porównań wykonanych przez obiekty $count\_str$ będące jego potomkami (tj. wytworzone przez dany obiekt klasy $benchar$ lub przez jego innych potomków)
    \end{itemize}
    \item Obiekt klasy $count\_str$:
    \begin{itemize}
        \item dziedziczy metody klasy $str$ (tj. standardowego łańcucha znaków w języku Python)
        \item posiada referencję do rodzica (tj. obiektu $benchar$, który go stworzył lub był rodzicem obiektu $count\_str$ który go stworzył)
        \item każda operacja porównująca znaki (m.in.: $==$, $!=$, $>$, $<$) odpowiednio zwiększa wartość pola $cmp\_count$ w rodzicu
        \item wszystkie operacje tworzące nowe łańcuchy znaków (m.in.: $+$, $[:]$) tworzą nowy obiekt $count\_str$, który dziedziczy rodzica.
    \end{itemize}
\end{itemize}

Przedstawione powyżej działanie klasy $benchar$ pozwala na skuteczne zliczanie porównań dla wykorzystanych implementacji algorytmów. Wynika to z tego, że wszystkie użyte implementacje działają, przyjmując tekst i wzorzec będące łańcuchami znaków (podczas pomiarów są to obiekty $count\_str$ wytworzone przez ten sam obiekt $benchar$) i podczas działania nie tworzą nowych łańcuchów znaków w inny sposób niż operacjami na samych łańcuchach (m.in.: $+$, $[:]$). Dzięki temu jest zagwarantowane, że wszystkie łańcuchy znaków wykorzystywane w danym uruchomieniu algorytmu będą odpowiednimi obiektami $count\_str$ i wszystkie wykonane porównania zostaną zliczone.

\subsection{Porównywane algorytmy}
W pomiarach liczby porównań znaków zostały sprawdzone następujące algorytmy:
\begin{itemize}
  \item BF -- algorytm naiwny
  \item MP -- algorytm Morissa i Pratta, czyli KMP wykorzystujący tablicę słabych prefikso-sufiksów
  \item KMP -- algorytm Knutha, Morissa i Pratta, czyli KMP wykorzystujący tablicę silnych prefikso-sufiksów
  \item BMB -- algorytm Boyera i Moore’a korzystający jedynie z tablicy $gss$ (bez $bcs$)
  \item BM -- algorytm Boyera i Moore’a korzystający z tablic $gss$ i $bcs$
  \item AGB -- algorytm Apostolico i Giancarlo korzystający jedynie z tablicy $gss$ (bez $bcs$)
  \item AG -- algorytm Apostolico i Giancarlo korzystający z tablic $gss$ i $bcs$
  \item Cr -- algorytm Crochemore'a dla uporządkowanych alfabetów
  \item GS -- algorytm Galila i Seiferasa
  \item TW -- algorytm Two-Way.
\end{itemize}

\section{Przeprowadzone testy}
W poniższym podrozdziale przedstawiono w formie wykresów wyniki pomiarów liczby porównań znaków wykonywanych przez dane algorytmy. Wykorzystane podczas pomiarów sposoby wyboru tekstu i wzorca można podzielić na dwie grupy. Pierwszą stanowią metody randomizowane (sekcje: rozkład jednostajny, rozkład geometryczny, język naturalny), dla których wspólną cechą w metodyce pomiarów było generowanie dla danych parametrów 100 par tekst-wzorzec i przedstawianie na wykresach średniej oraz maksymalnej liczby wykonanych porównań w tej próbce. Drugą grupę stanowią trudne przypadki świadczące o pesymistycznej liczbie porównań wykonywanej przez część algorytmów (sekcje: pesymistyczny przypadek dla algorytmu naiwnego, pesymistyczny przypadek dla algorytmu BM, pesymistyczny przypadek dla algorytmu AG). W ich przypadkach dla danych parametrów wykonano (i zaprezentowano na wykresach) jeden pomiar, gdyż są to jednoznacznie ustalone wzorce i słowa.

\FloatBarrier
\subsection{Rozkład jednostajny}
W tej sekcji przedstawiono wyniki pomiarów dla tekstów i wzorców generowanych zgodnie z rozkładem jednostajnym nad ustalonym alfabetem (każda litera ma prawdopodobieństwo równe $\frac{1}{|\mathcal{A}|}$ wystąpienia na danym indeksie tekstu oraz wzorca).

\begin{figure}[htb]
    \centering
    \begin{subfigure}[b]{0.49\textwidth}
        \centering
        \includesvg[width=\textwidth]{result_alph_uni_3_pat_10_plot_avg_100}
        \caption{średnia liczba porównań}
        \label{fig:result_alph_uni_3_pat_10_plot_avg_100}
    \end{subfigure}
    \hfill
    \begin{subfigure}[b]{0.49\textwidth}
        \centering
        \includesvg[width=\textwidth]{result_alph_uni_3_pat_10_plot_max_100}
        \caption{maksymalna liczba porównań}
        \label{fig:result_alph_uni_3_pat_10_plot_max_100}
    \end{subfigure}
    \hfill
    \caption{Liczba porównań w zależności od $len(text)$ dla $|\mathcal{A}| = 3$ i $len(pat) = 10$}
    \label{fig:result_alph_uni_3_pat_10}
\end{figure}

\begin{figure}[htb]
    \centering
    \begin{subfigure}[b]{0.49\textwidth}
        \centering
        \includesvg[width=\textwidth]{result_alph_uni_3_pat_len(text)2_plot_avg_100}
        \caption{średnia liczba porównań}
        \label{fig:result_alph_uni_3_pat_len(text)2_plot_avg_100}
    \end{subfigure}
    \hfill
    \begin{subfigure}[b]{0.49\textwidth}
        \centering
        \includesvg[width=\textwidth]{result_alph_uni_3_pat_len(text)2_plot_max_100}
        \caption{maksymalna liczba porównań}
        \label{fig:result_alph_uni_3_pat_len(text)2_plot_max_100}
    \end{subfigure}
    \hfill
    \caption{Liczba porównań w zależności od $len(text)$ dla $|\mathcal{A}| = 3$ i $len(pat) = \frac{len(text)}{2}$}
    \label{fig:result_alph_uni_3_pat_len(text)2}
\end{figure}

\begin{figure}[htb]
    \centering
    \begin{subfigure}[b]{0.49\textwidth}
        \centering
        \includesvg[width=\textwidth]{result_alph_uni_3_text_1000_plot_avg_100}
        \caption{średnia liczba porównań}
        \label{fig:result_alph_uni_3_text_1000_plot_avg_100}
    \end{subfigure}
    \hfill
    \begin{subfigure}[b]{0.49\textwidth}
        \centering
        \includesvg[width=\textwidth]{result_alph_uni_3_text_1000_plot_max_100}
        \caption{maksymalna liczba porównań}
        \label{fig:result_alph_uni_3_text_1000_plot_max_100}
    \end{subfigure}
    \hfill
    \caption{Liczba porównań w zależności od $len(pat)$ dla $|\mathcal{A}| = 3$ i $len(text) = 1000$}
    \label{fig:result_alph_uni_3_text_1000}
\end{figure}

\begin{figure}[htb]
    \centering
    \begin{subfigure}[b]{0.49\textwidth}
        \centering
        \includesvg[width=\textwidth]{result_alph_uni_26_pat_10_plot_avg_100}
        \caption{średnia liczba porównań}
        \label{fig:result_alph_uni_26_pat_10_plot_avg_100}
    \end{subfigure}
    \hfill
    \begin{subfigure}[b]{0.49\textwidth}
        \centering
        \includesvg[width=\textwidth]{result_alph_uni_26_pat_10_plot_max_100}
        \caption{maksymalna liczba porównań}
        \label{fig:result_alph_uni_26_pat_10_plot_max_100}
    \end{subfigure}
    \hfill
    \caption{Liczba porównań w zależności od $len(text)$ dla $|\mathcal{A}| = 26$ i $len(pat) = 10$}
    \label{fig:result_alph_uni_26_pat_10}
\end{figure}

\begin{figure}[htb]
    \centering
    \begin{subfigure}[b]{0.49\textwidth}
        \centering
        \includesvg[width=\textwidth]{result_alph_uni_26_pat_len(text)2_plot_avg_100}
        \caption{średnia liczba porównań}
        \label{fig:result_alph_uni_26_pat_len(text)2_plot_avg_100}
    \end{subfigure}
    \hfill
    \begin{subfigure}[b]{0.49\textwidth}
        \centering
        \includesvg[width=\textwidth]{result_alph_uni_26_pat_len(text)2_plot_max_100}
        \caption{maksymalna liczba porównań}
        \label{fig:result_alph_uni_26_pat_len(text)2_plot_max_100}
    \end{subfigure}
    \hfill
    \caption{Liczba porównań w zależności od $len(text)$ dla $|\mathcal{A}| = 26$ i $len(pat) = \frac{len(text)}{2}$}
    \label{fig:result_alph_uni_26_pat_len(text)2}
\end{figure}

\begin{figure}[htb]
    \centering
    \begin{subfigure}[b]{0.49\textwidth}
        \centering
        \includesvg[width=\textwidth]{result_alph_uni_26_text_1000_plot_avg_100}
        \caption{średnia liczba porównań}
        \label{fig:result_alph_uni_26_text_1000_plot_avg_100}
    \end{subfigure}
    \hfill
    \begin{subfigure}[b]{0.49\textwidth}
        \centering
        \includesvg[width=\textwidth]{result_alph_uni_26_text_1000_plot_max_100}
        \caption{maksymalna liczba porównań}
        \label{fig:result_alph_uni_26_text_1000_plot_max_100}
    \end{subfigure}
    \hfill
    \caption{Liczba porównań w zależności od $len(pat)$ dla $|\mathcal{A}| = 26$ i $len(text) = 1000$}
    \label{fig:result_alph_uni_26_text_1000}
\end{figure}

\FloatBarrier
\subsection{Rozkład geometryczny}
W tej sekcji przedstawiono wyniki pomiarów dla tekstów i wzorców generowanych zgodnie z rozkładem geometrycznym z ustalonym parametrem $p$ ($i$-ta litera alfabetu ma prawdopodobieństwo równe $(1-p)^{i-1}p$ wystąpienia na danym indeksie tekstu oraz wzorca).

\begin{figure}[htb]
    \centering
    \begin{subfigure}[b]{0.49\textwidth}
        \centering
        \includesvg[width=\textwidth]{result_alph_geo_02_pat_10_plot_avg_100}
        \caption{średnia liczba porównań}
        \label{fig:result_alph_geo_02_pat_10_plot_avg_100}
    \end{subfigure}
    \hfill
    \begin{subfigure}[b]{0.49\textwidth}
        \centering
        \includesvg[width=\textwidth]{result_alph_geo_02_pat_10_plot_max_100}
        \caption{maksymalna liczba porównań}
        \label{fig:result_alph_geo_02_pat_10_plot_max_100}
    \end{subfigure}
    \hfill
    \caption{Liczba porównań w zależności od $len(text)$ dla $p = 0.2$ i $len(pat) = 10$}
    \label{fig:result_alph_geo_02_pat_10}
\end{figure}

\begin{figure}[htb]
    \centering
    \begin{subfigure}[b]{0.49\textwidth}
        \centering
        \includesvg[width=\textwidth]{result_alph_geo_02_pat_len(text)2_plot_avg_100}
        \caption{średnia liczba porównań}
        \label{fig:result_alph_geo_02_pat_len(text)2_plot_avg_100}
    \end{subfigure}
    \hfill
    \begin{subfigure}[b]{0.49\textwidth}
        \centering
        \includesvg[width=\textwidth]{result_alph_geo_02_pat_len(text)2_plot_max_100}
        \caption{maksymalna liczba porównań}
        \label{fig:result_alph_geo_02_pat_len(text)2_plot_max_100}
    \end{subfigure}
    \hfill
    \caption{Liczba porównań w zależności od $len(text)$ dla $p = 0.2$ i $len(pat) = \frac{len(text)}{2}$}
    \label{fig:result_alph_geo_02_pat_len(text)2}
\end{figure}

\begin{figure}[htb]
    \centering
    \begin{subfigure}[b]{0.49\textwidth}
        \centering
        \includesvg[width=\textwidth]{result_alph_geo_02_text_1000_plot_avg_100}
        \caption{średnia liczba porównań}
        \label{fig:result_alph_geo_02_text_1000_plot_avg_100}
    \end{subfigure}
    \hfill
    \begin{subfigure}[b]{0.49\textwidth}
        \centering
        \includesvg[width=\textwidth]{result_alph_geo_02_text_1000_plot_max_100}
        \caption{maksymalna liczba porównań}
        \label{fig:result_alph_geo_02_text_1000_plot_max_100}
    \end{subfigure}
    \hfill
    \caption{Liczba porównań w zależności od $len(pat)$ dla $p = 0.2$ i $len(text) = 1000$}
    \label{fig:result_alph_geo_02_text_1000}
\end{figure}

\begin{figure}[htb]
    \centering
    \begin{subfigure}[b]{0.49\textwidth}
        \centering
        \includesvg[width=\textwidth]{result_alph_geo_05_pat_10_plot_avg_100}
        \caption{średnia liczba porównań}
        \label{fig:result_alph_geo_05_pat_10_plot_avg_100}
    \end{subfigure}
    \hfill
    \begin{subfigure}[b]{0.49\textwidth}
        \centering
        \includesvg[width=\textwidth]{result_alph_geo_05_pat_10_plot_max_100}
        \caption{maksymalna liczba porównań}
        \label{fig:result_alph_geo_05_pat_10_plot_max_100}
    \end{subfigure}
    \hfill
    \caption{Liczba porównań w zależności od $len(text)$ dla $p = 0.5$ i $len(pat) = 10$}
    \label{fig:result_alph_geo_05_pat_10}
\end{figure}

\begin{figure}[htb]
    \centering
    \begin{subfigure}[b]{0.49\textwidth}
        \centering
        \includesvg[width=\textwidth]{result_alph_geo_05_pat_len(text)2_plot_avg_100}
        \caption{średnia liczba porównań}
        \label{fig:result_alph_geo_05_pat_len(text)2_plot_avg_100}
    \end{subfigure}
    \hfill
    \begin{subfigure}[b]{0.49\textwidth}
        \centering
        \includesvg[width=\textwidth]{result_alph_geo_05_pat_len(text)2_plot_max_100}
        \caption{maksymalna liczba porównań}
        \label{fig:result_alph_geo_05_pat_len(text)2_plot_max_100}
    \end{subfigure}
    \hfill
    \caption{Liczba porównań w zależności od $len(text)$ dla $p = 0.5$ i $len(pat) = \frac{len(text)}{2}$}
    \label{fig:result_alph_geo_05_pat_len(text)2}
\end{figure}

\begin{figure}[htb]
    \centering
    \begin{subfigure}[b]{0.49\textwidth}
        \centering
        \includesvg[width=\textwidth]{result_alph_geo_05_text_1000_plot_avg_100}
        \caption{średnia liczba porównań}
        \label{fig:result_alph_geo_05_text_1000_plot_avg_100}
    \end{subfigure}
    \hfill
    \begin{subfigure}[b]{0.49\textwidth}
        \centering
        \includesvg[width=\textwidth]{result_alph_geo_05_text_1000_plot_max_100}
        \caption{maksymalna liczba porównań}
        \label{fig:result_alph_geo_05_text_1000_plot_max_100}
    \end{subfigure}
    \hfill
    \caption{Liczba porównań w zależności od $len(pat)$ dla $p = 0.5$ i $len(text) = 1000$}
    \label{fig:result_alph_geo_05_text_1000}
\end{figure}

\begin{figure}[htb]
    \centering
    \begin{subfigure}[b]{0.49\textwidth}
        \centering
        \includesvg[width=\textwidth]{result_alph_geo_08_pat_10_plot_avg_100}
        \caption{średnia liczba porównań}
        \label{fig:result_alph_geo_08_pat_10_plot_avg_100}
    \end{subfigure}
    \hfill
    \begin{subfigure}[b]{0.49\textwidth}
        \centering
        \includesvg[width=\textwidth]{result_alph_geo_08_pat_10_plot_max_100}
        \caption{maksymalna liczba porównań}
        \label{fig:result_alph_geo_08_pat_10_plot_max_100}
    \end{subfigure}
    \hfill
    \caption{Liczba porównań w zależności od $len(text)$ dla $p = 0.8$ i $len(pat) = 10$}
    \label{fig:result_alph_geo_08_pat_10}
\end{figure}

\begin{figure}[htb]
    \centering
    \begin{subfigure}[b]{0.49\textwidth}
        \centering
        \includesvg[width=\textwidth]{result_alph_geo_08_pat_len(text)2_plot_avg_100}
        \caption{średnia liczba porównań}
        \label{fig:result_alph_geo_08_pat_len(text)2_plot_avg_100}
    \end{subfigure}
    \hfill
    \begin{subfigure}[b]{0.49\textwidth}
        \centering
        \includesvg[width=\textwidth]{result_alph_geo_08_pat_len(text)2_plot_max_100}
        \caption{maksymalna liczba porównań}
        \label{fig:result_alph_geo_08_pat_len(text)2_plot_max_100}
    \end{subfigure}
    \hfill
    \caption{Liczba porównań w zależności od $len(text)$ dla $p = 0.8$ i $len(pat) = \frac{len(text)}{2}$}
    \label{fig:result_alph_geo_08_pat_len(text)2}
\end{figure}

\begin{figure}[htb]
    \centering
    \begin{subfigure}[b]{0.49\textwidth}
        \centering
        \includesvg[width=\textwidth]{result_alph_geo_08_text_1000_plot_avg_100}
        \caption{średnia liczba porównań}
        \label{fig:result_alph_geo_08_text_1000_plot_avg_100}
    \end{subfigure}
    \hfill
    \begin{subfigure}[b]{0.49\textwidth}
        \centering
        \includesvg[width=\textwidth]{result_alph_geo_08_text_1000_plot_max_100}
        \caption{maksymalna liczba porównań}
        \label{fig:result_alph_geo_08_text_1000_plot_max_100}
    \end{subfigure}
    \hfill
    \caption{Liczba porównań w zależności od $len(pat)$ dla $p = 0.8$ i $len(text) = 1000$}
    \label{fig:result_alph_geo_08_text_1000}
\end{figure}

\FloatBarrier
\subsection{Język naturalny}
W tej sekcji przedstawiono wyniki pomiarów dla tekstów i wzorców wybieranych losowo z pewnego tekstu języka naturalnego (Adam Mickiewicz: "Pan Tadeusz", źródło: \url{https://gist.github.com/tomekziel/47b7c3dffb17adbc70c42b9be8dc93a7}), który został znormalizowany poprzez zamianę wszystkich liter na małe, zamianę wszystkich znaków białych na spacje i usunięcie znaków niebędących znakami alfanumerycznymi lub spacjami.

\begin{figure}[htb]
    \centering
    \begin{subfigure}[b]{0.49\textwidth}
        \centering
        \includesvg[width=\textwidth]{result_natural_pat_10_plot_avg_100}
        \caption{średnia liczba porównań}
        \label{fig:result_natural_pat_10_plot_avg_100}
    \end{subfigure}
    \hfill
    \begin{subfigure}[b]{0.49\textwidth}
        \centering
        \includesvg[width=\textwidth]{result_natural_pat_10_plot_max_100}
        \caption{maksymalna liczba porównań}
        \label{fig:result_natural_pat_10_plot_max_100}
    \end{subfigure}
    \hfill
    \caption{Liczba porównań w zależności od $len(text)$ dla $len(pat) = 10$}
    \label{fig:result_natural_pat_10}
\end{figure}

\begin{figure}[htb]
    \centering
    \begin{subfigure}[b]{0.49\textwidth}
        \centering
        \includesvg[width=\textwidth]{result_natural_pat_len(text)2_plot_avg_100}
        \caption{średnia liczba porównań}
        \label{fig:result_natural_pat_len(text)2_plot_avg_100}
    \end{subfigure}
    \hfill
    \begin{subfigure}[b]{0.49\textwidth}
        \centering
        \includesvg[width=\textwidth]{result_natural_pat_len(text)2_plot_max_100}
        \caption{maksymalna liczba porównań}
        \label{fig:result_natural_pat_len(text)2_plot_max_100}
    \end{subfigure}
    \hfill
    \caption{Liczba porównań w zależności od $len(text)$ dla $len(pat) = \frac{len(text)}{2}$}
    \label{fig:result_natural_pat_len(text)2}
\end{figure}

\begin{figure}[htb]
    \centering
    \begin{subfigure}[b]{0.49\textwidth}
        \centering
        \includesvg[width=\textwidth]{result_natural_text_1000_plot_avg_100}
        \caption{średnia liczba porównań}
        \label{fig:result_natural_text_1000_plot_avg_100}
    \end{subfigure}
    \hfill
    \begin{subfigure}[b]{0.49\textwidth}
        \centering
        \includesvg[width=\textwidth]{result_natural_text_1000_plot_max_100}
        \caption{maksymalna liczba porównań}
        \label{fig:result_natural_text_1000_plot_max_100}
    \end{subfigure}
    \hfill
    \caption{Liczba porównań w zależności od $len(pat)$ dla $len(text) = 1000$}
    \label{fig:result_natural_text_1000}
\end{figure}

\FloatBarrier
\subsection{Pesymistyczny przypadek dla algorytmu naiwnego}
W poniższej sekcji przedstawiono wyniki dla tekstów i wzorców będących przypadkiem pesymistycznym dla algorytmu naiwnego, czyli $text = a^{e}b$ i $pat = a^{\frac{e}{2}}b$, gdzie $a$ i $b$ są różnymi literami.

\begin{figure}[htb]
    \centering
    \includesvg[width=0.49\textwidth]{result_bf_hard_plot}
    \hfill
    \caption{Liczba porównań w zależności od $len(text)$}
    \label{fig:result_bf_hard_plot}
\end{figure}

\FloatBarrier
\subsection{Pesymistyczny przypadek dla algorytmu BM}
W poniższej sekcji przedstawiono wyniki dla tekstów i wzorców będących przypadkiem pesymistycznym dla algorytmu BM, czyli $text = a^{n}$ i $pat = a^{m}$, gdzie $a$ jest literą.

\begin{figure}[htb]
    \centering
    \includesvg[width=0.49\textwidth]{result_bm_hard_pat_10_plot}
    \hfill
    \caption{Liczba porównań w zależności od $len(text)$ dla $len(pat) = 10$}
    \label{fig:result_bm_hard_pat_10_plot}
\end{figure}

\begin{figure}[htb]
    \centering
    \includesvg[width=0.49\textwidth]{result_bm_hard_pat_len(text)2_plot}
    \hfill
    \caption{Liczba porównań w zależności od $len(text)$ dla $len(pat) = \frac{len(text)}{2}$}
    \label{fig:result_bm_hard_pat_len(text)2_plot}
\end{figure}

\begin{figure}[htb]
    \centering
    \includesvg[width=0.49\textwidth]{result_bm_hard_text_1000_plot}
    \hfill
    \caption{Liczba porównań w zależności od $len(pat)$ dla $len(text) = 1000$}
    \label{fig:result_bm_hard_text_1000_plot}
\end{figure}

\FloatBarrier
\subsection{Pesymistyczny przypadek dla algorytmu AG}
W poniższej sekcji przedstawiono wyniki dla tekstów i wzorców będących przypadkiem pesymistycznym dla algorytmu AG, czyli $text = pat^{e}$ i $pat = a^{m-1}ba^{m}b$, gdzie $a$ i $b$ są różnymi literami.

\begin{figure}[htb]
    \centering
    \includesvg[width=0.49\textwidth]{result_ag_hard_pat_10_plot}
    \hfill
    \caption{Liczba porównań w zależności od $len(text)$ dla $len(pat) = 10$}
    \label{fig:result_ag_hard_pat_10_plot}
\end{figure}

\begin{figure}[htb]
    \centering
    \includesvg[width=0.49\textwidth]{result_ag_hard_pat_len(text)2_plot}
    \hfill
    \caption{Liczba porównań w zależności od $len(text)$ dla $len(pat) = \frac{len(text)}{2}$}
    \label{fig:result_ag_hard_pat_len(text)2_plot}
\end{figure}

\FloatBarrier
\newpage
\section{Wnioski}
Na podstawie powyższych pomiarów można stwierdzić, że \textbf{w celu wyboru najwydajniejszego algorytmu dla danego zastosowania powinno się wziąć pod uwagę przede wszystkim rozmiar alfabetu i długość wzorca względem długości tekstu}. 

Można zauważyć jednak, że \textbf{w większości przypadków najmniejszą liczbę porównań (do uzyskania której dążymy) wykona jedna z wersji algorytmu BM} -- często liczba porównań będzie nawet mniejsza niż długość tekstu. \textbf{W przypadku małego rozmiaru alfabetu lepiej sprawdza się wersja BMB} -- pomijająca przesunięcie nieodpowiedniego znaku (rys. \ref{fig:result_alph_uni_3_pat_10}--\ref{fig:result_alph_uni_3_text_1000} i rys. \ref{fig:result_alph_geo_08_pat_10}--\ref{fig:result_alph_geo_08_text_1000}). W takich przypadkach korzyść z użycia tej tablicy jest znikoma, a obliczenie jej jest pewnym kosztem. 

Sytuacja zmienia się znacznie w przypadku dużych alfabetów -- w szczególności złożonego z wszystkich liter alfabetu łacińskiego. W takich przypadkach przesunięcia dobrego sufiksu są dominowane przez przesunięcia nieodpowiedniego znaku. Różnica w liczbie porównań jest tym większa, im krótszy jest wzorzec, gdyż koszt dodatkowego preprocessingu jest wtedy odpowiednio mniejszy (rys. \ref{fig:result_alph_uni_26_pat_10}--\ref{fig:result_alph_geo_02_pat_10} i rys. \ref{fig:result_alph_geo_05_pat_10}). \textbf{Dla dużych alfabetów wydajniejsza jest więc pełna wersja algorytmu BM}.

Warto zauważyć, że spośród przeprowadzonych testów \textbf{tylko w jednym przypadku algorytm AG wykonał mniej porównań niż algorytm BM (i odpowiednio AGB niż BMB) - w przypadku pesymistycznym dla algorytmu BM} (rys. \ref{fig:result_bm_hard_pat_10_plot}--\ref{fig:result_bm_hard_text_1000_plot}). Warto jednak zauważyć, że różnice w wydajności dla pozostałych przypadków nie były zwykle znaczne, więc warto rozważyć, czy dla danego zastosowania ważniejszy jest średni zysk wydajności, czy gwarancja liniowego ograniczenia liczby wykonanych porównań.

Na koniec zwróćmy uwagę na jeszcze jeden ciekawy wynik. \textbf{Dla odpowiednio długich wzorców, w przypadku wyszukiwania w tekście naturalnym, najlepiej sprawdzał się algorytm naiwny} (rys. \ref{fig:result_natural_pat_len(text)2}). Jest to zapewne spowodowane tym, że pesymistyczne przypadki dla algorytmu naiwnego (czyli długie słowa mające krótki najkrótszy cykl) rzadko występują w języku naturalnym. Algorytm naiwny zyskuje przewagę, gdyż nie wykonuje preprocessingu, podczas którego liczba wykonanych porównań jest uzależniona od długości wzorca.