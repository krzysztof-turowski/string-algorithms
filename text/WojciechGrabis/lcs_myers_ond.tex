\documentclass[12pt]{article}
\usepackage[T1]{fontenc}
\usepackage[utf8]{inputenc}
\usepackage{amsmath}
\usepackage{polski}
\usepackage{amsthm}
\usepackage{algorithm}
\usepackage{algorithmic}
\usepackage{mathtools}
\DeclarePairedDelimiter\ceil{\lceil}{\rceil}
\DeclarePairedDelimiter\floor{\lfloor}{\rfloor}

\renewcommand{\thealgorithm}{}

\newtheorem*{definition}{Definicja}
\newtheorem*{lemma}{Lemat}

\title{LCS w czasie oczekiwanym O(ND)}
\author{Wojciech Grabis}
\date{June 2020}


\begin{document}

\maketitle

\section{Wprowadzenie}

Algorytm omówiony w pracy służy do poszukiwania najdłuższego wspólnego podciągu. Powstał w oparciu o pracę \textbf{An O(ND) Diffrence Algorithm and Its Variations} autorstwa \textbf{Eugune W.Myers}.
\section{Definicje}
Zanim przejdziemy do algorytmu zdefiniujmy graf edycji, definicje wspólnego podciągu oraz pewne pomocnicze określenia.

\subsection{Graf edycji}
 Mając dane słowa $A = a_{1}a_{2}\dots a_{N}$ oraz $B = b_{1}b_{2}\dots b_{M}$ jako graf edycji G oznaczamy skierowany graf, dla którego zbiór wierzchołków to punkty siatki $(x,y),\:x\in[0,N]\:i\:y\in[0,M]$, natomiast na krawędzi składają się:
\begin{itemize}
    \item $(x-1,y) \xrightarrow{}(x,y)$ dla każdego $x\in[1,N]$ oraz $y\in[0,M]$.
    \item $(x,y-1) \xrightarrow{}(x,y)$ dla każdego $x\in[0,N]$ oraz $y\in[1,M]$.
    \item $(x-1,y-1) \xrightarrow{}(x,y)$ jeśli $a_{x} = b_{y}$, taka krawędź definiuje pewne dopasowanie pomiędzy wzorcami, punkty dla których $a_{x} = b_{y}$ nazwiemy punktami dopasowania.
\end{itemize}

\begin{definition}[Podciąg słowa]

Podciąg słowa A to każde słowo powstałe poprzez usunięcie 0 lub więcej symboli z A. Natomiast wspólny podciąg dwóch słów A i B to podciąg każdego z nich. 
\end{definition}

\subsection{Graf edycji, a wspólny podciąg}

\begin{definition}[Ślad]
Śladem długości L nazywamy sekwencje L punktów dopasowania $(x_{1},y_{1}),(x_{2},y_{2})\dots(x_{L}, y_{L})$, gdzie dla każdego $1 \leq i < L:\:x_{i}<x_{i+1} \wedge y_{i}<y_{i+1}$. Jest to pewna sekwencji diagonalnych krawędzi. Ślad możemy powiązać z pewną ścieżka pomiędzy $(0,0)$ a $(N, M)$.
\end{definition}

Każdy ślad grafu edycji definiuje pewien wspólny podciąg słów A i B, który powstaje poprzez odpowiednie symbole zdefiniowane przez punkty dopasowania.

\begin{definition}[Odległość edycyjna]
Odległość edycyjna D to ilość operacji edycji które przeprowadzamy na tekście A żeby przeprowadzić go w tekst B. Jeśli spojrzymy na ścieżkę która zawiera pewien ślad L to każda krawędź pozioma w tej ścieżce definiuje pewną operację usunięcia symbolu z tekstu A, natomiast krawędzi pionowe definiują operację dodania symboli do tekstu A za odpowiednim indeksem w ciągu A. Możemy zauważyć że wartość odległości edycyjnej jest powiązana z długością śladu L, tj $D = N + M - 2L$.
\end{definition}

Idea algorytmu będzie opierała się na wyszukiwaniu najdłuższego śladu pomiędzy punktami (0,0) oraz (N,M), czyli znalezieniu ścieżki która zawiera największą ilość krawędzi diagonalnych, powyższy ślad wyznaczy nam najdłuższy wspólny podciąg słów A i B.

\vspace{2mm}

Pozostały nam jeszcze dwie definicję, które będą wykorzystywane w algorytmie:
\begin{definition}[D-ścieżka]
Jako D-ścieżkę definiujemy ścieżkę w grafie edycji rozpoczynającą się w punkcie $(0,0)$, posiadającą dokładnie D krawędzi niediagonalnych.
\end{definition}

\begin{definition}[k-diagonala]
W celu wprowadzenia algorytmu zdefiniujemy numerację diagonali. K-diagonala, to taka krawędź diagonalna, która składa się z punktów $(x,y)$, takich że $x - y = k$. Zgodnie z tym krawędzi są numerowane od $-M$ do $N$.
\end{definition}

Na podstawię definicji k-diagonali zauważmy, że krawędź pionowa (pozioma) której punkt startowy leży na k-diagonali, posiada drugi koniec krawędzi na (k-1)-diagonali (odpowiednio (k+1)-diagonali dla poziomej). Zbiór diagonali znajdujących się na końcu krawędzi pionowej/poziomej nazywamy ogonem (w przypadku gdy koniec krawędzi nie jest początkiem żadnej diagonali, wtedy ogon nazywamy pustym).

\section{Algorytm}
\subsection{Poszukiwanie najdłuższej ścieżki}
Algorytm będzie opierał się na wyszukiwaniu najdalszych D-ścieżek w k-diagonali. Przed wprowadzeniem pseudokodu, najpierw omówimy pewne właściwości związane z diagonalami oraz ścieżkami

\begin{lemma}

Każda D-ścieżka musi kończyć się na k-diagonali, takiej że $k \in \{-D, -D + 2, \dots, D-2, D\}$
\end{lemma}

\begin{proof}
Dowód indukcyjny. Dla D = 0 zauważmy, że 0-ścieżka może kończyć się tylko na diagonali 0, ponieważ oba indeksy $(x,y)$ na tej ścieżce muszą być sobie równe (nie możemy wykorzystać żadnej krawędzi poziomej lub pionowej z definicji 0-ścieżki). Następnie zauważmy, że D-ścieżka składa się z pewnej (D-1)-ścieżki, pewnej poziomej/pionowej krawędzi oraz ogona. Z założenia indukcji mamy że ścieżka D-1 kończy się na pewnej k krawędzi gdzie $k \in \{ -D+1, -D + 3, \dots D -3, D - 1\}$, natomiast nowy ogon będzie składał się z pewnej krawędzi $k \pm 1$(bazując na obserwacji z k-diagonalami), stąd zbiór diagonali na których kończy się D-ścieżka to $k_{D} \in \{ -D + 1 \pm 1, -D + 3\pm 1, \dots D -3\pm 1, D - 1\pm 1\}$, czyli mamy $k_{D} \in \{ -D, -D + 2, \dots, D - 2, D\}$.
\end{proof}

\begin{definition}[Najdalej sięgająca D-ścieżka]
D-ścieżkę nazywamy najdalej sięgającą w diagonali k, jeśli jest to ścieżka której końcowy punkt ma największy numer kolumny (odpowiednio numer wiersza z definicji k-diagonali) spośród wszystkich D-ścieżek w k diagonali.
\end{definition}

\begin{lemma}[Rozbicie najdalej sięgającej D-ścieżki]
Najdalej sięgająca D-ścieżka w diagonali k może być rozbita na najdalej sięgającą (D-1)-ścieżkę kończącą się na diagonali k - 1, krawędź poziomą oraz najdłuższy ogon z tej krawędzi, lub w najdalej sięgającą (D-1)-ścieżkę w diagonali k + 1, krawędź pionową oraz najdłuższy ogon z tej krawędzi.
\end{lemma}

\begin{proof}
Wybranie odpowiedniej najdalszej (D-1)-ścieżki w diagonali k+1/k-1 związane jest ze wcześniej wspomnianą obserwacją na temat ogona krawędzi poziomej/pionowej. Natomiast w celu udowodnienie faktu że to najdalej sięgająca (D-1)-ścieżka załóżmy że (D-1)-ścieżka składająca się na kompozycję najdalej sięgającej D-ścieżki nie jest najdalej sięgająca w diagonali k, ale możemy zaobserwować że skoro D-ścieżka musi sięgać dalej niż (D-1)-ścieżka, to oznacza że do ogona tej D-ścieżki będziemy mogli się też dostać z końca ogona najdalszej (D-1)-ścieżki przy pomocy odpowiedniej krawędzi niediagonalnej, więc zawsze możemy najdalszą D-ścieżkę rozłożyć na najdalszą (D-1)-ścieżkę.
\end{proof}

Na bazie obu lematów wprowadzamy algorytm obliczania punktu końca najdalej sięgającej D-ścieżki w diagonali k. Zauważmy że odległość edycyjna $D \in [0, M+N]$, więc będziemy kolejno badali możliwe odległości edycyjne, aż dojdziemy do punktu $(N,M)$.

\begin{algorithm}
\caption{Obliczanie odległości edycyjnej}
\begin{algorithmic}
\STATE $V:[0] + [0] * 2(m+n)$
\STATE $V[1] \leftarrow 0$
\FOR{$D \leftarrow 0$ \TO $M+N$}
    \FOR{$k \leftarrow -D$ \TO $D$ in steps of 2}
        \IF{$k= -D$ or $k \neq D$ and $V[k-1] < V[k+1]$}
            \STATE $x \leftarrow V[k+1]$
        \ELSE
            \STATE $x \leftarrow V[k-1] + 1$
        \ENDIF
        \STATE $y \leftarrow x-k$
        \WHILE{$x<N$ and $y<M$ and $a_{x+1}=b_{y+1}$}
            \STATE $V[k] \leftarrow x$ 
        \ENDWHILE
        \IF{$x \geq N$ and $y \geq M$}
            \RETURN D
        \ENDIF
    \ENDFOR
\ENDFOR
\end{algorithmic}
\end{algorithm}

\vspace{5mm}

Powyższy algorytm pozwala nam w czasie O((M + N)D) wyznaczyć odległość edycyjną (a co za tym idzie również długość najdłuższego wspólnego podciągu) przy wykorzystaniu O(D) pamięci.

Niestety problem pojawia się przy wyznaczeniu symboli na które składa się najdłuższy wspólny podciąg. Musielibyśmy po każdej iteracji zapisywać tablicę V, w celu wyznaczenia z których punktów przechodziliśmy w tej ścieżce, co sprawiałoby że musielibyśmy wykorzystać $O(D^2)$ pamięci.

W tym celu w następnej części omówimy algorytm bazujący na przeszukiwaniu najdalszych ścieżek, który będzie działał w O(D) pamięci.

\subsection{Algorytm w liniowej pamięci}

Finalny algorytm na poszukiwanie najdłuższego wspólnego podciągu będzie opierał się na strategii dziel i rządź. Algorytm będzie korzystał z poszukiwania najdalej sięgającej ścieżki z punktu $(0,0)$ oraz odwrotnej ścieżki z punktu $(N,M)$. W celu obliczenia odwrotnej ścieżki będziemy korzystać z poprzedniego algorytmu, z tym że tym razem będziemy szukali ścieżki z $(N,M)$ w kierunku $(0,0)$ odwracając krawędzie z grafu edycji oraz minimalizując wartość Y w przeciwieństwie do maksymalizowania X z poprzedniego algorytmu (natomiast ścieżka w przód liczona jest analogicznie do poprzedniego algorytmu). Przed zaprezentowaniem algorytmu udowodnimy następujący lemat:

\begin{lemma}{Podział D-ścieżek}
Pomiędzy punktami $(0,0)$ i $(N, M)$ istnieje D-ścieżka wtedy i tylko wtedy gdy istnieje $\ceil*{D/2}$-ścieżka z $(0,0)$ do punktu $(x,y)$ oraz $\floor*{D/2}$-ścieżka z punktu $(u,v)$ do $(M,N)$, taka że
\begin{itemize}
    \item $u+v \geq \ceil*{D/2}$ oraz $x+y \leq N + M - \floor*{D/2}$
    \item $x-y = u-v$ oraz $x \geq u$.
\end{itemize}
\end{lemma}

Generalnie powyższy lemat opiera się na podziale D-ścieżki na pewne 2 ścieżki, jedna ścieżka z $(0,0)$ a druga ścieżka, którą możemy traktować jako odwrotną ścieżkę z $(N,M)$ w kierunku $(0,0)$. Zauważmy że jak rozłożymy D-ścieżkę na 2 takie przeciwne ścieżki, to ogon znajdujący się na końcu ścieżki do przodu będzie pokrywał się z ogonem znajdującym się na końcu ścieżki przeciwnej, będziemy właśnie z tej właściwości korzystać szukając ścieżkę do przodu i ścieżkę od końca, aż nasze ścieżki pokryją się w pewnej diagonali. Właśnie te diagonale z ogona które się pokrywają wyznaczą nam symbole które będą "środkiem" naszego najdłuższego wspólnego podciągu. 

Następnie wykorzystamy strategię dziel i rządź rozdzielimy sobie słowa na podstawie środkowego ogona i wykonamy się rekurencyjnie na naszych podsłowach, największy wspólny podciąg to rekurencyjne wywołanie lewej części słów, znaków z ogona oraz wywołania na prawej części słów.

\vspace{5mm}

Algorytm będzie opierał się na odpowiedniej adaptacji poprzedniego algorytmu do przeszukiwania wprzód D-ścieżek oraz wstecznych D-ścieżek, w przypadku gdy dla jakiejś diagonali k ogony ścieżek się pokryją zwracamy odpowiednią odległość edycyjną oraz ogon który znaleźliśmy. W przypadku gdy dla jakiegoś D, D-ścieżka wprzód pokrywa się z (D-1)ścieżką przeciwną, to zwracamy $2D - 1$, natomiast gdy D-ścieżka przeciwna pokrywa się z D-ścieżką wprzód zwracamy $2D$.

\vspace{2mm}

Nasz finalny algorytm na znalezienie największego wspólnego podciągu prezentuję się następująco:

\begin{algorithm}
\caption{LCS(A,B,N,M)}
\begin{algorithmic}
\IF{$N > 0$ and $M > 0$}
    \STATE{ Uruchamiamy naszą procedurę szukania środkowego ogona} 
    \STATE{$(X,Y) \xrightarrow{} (U,V)$ to nasz środkowy ogon, D - odległość edycyjna }
    \IF {$D > 1$}
        \STATE $left = LCS(A[1\dots x],B[1\dots y],x,y)$
        \STATE $middle = A[x+1\dots u]$
        \STATE $right = LCS(A[u+1\dots N],B[v+1\dots M],N-u,M-v)$
        \RETURN $left + middle + right$
    \ELSIF {$M > N$}
        \RETURN $A[1\dots N]$
    \ELSE
        \RETURN $B[1\dots M]$
    \ENDIF
\ENDIF
\end{algorithmic}
\end{algorithm}

\subsection{Złożoność}

Oznaczmy jako $T(P,D)$ czas działania algorytmu gdzie $P = N + M$. Nasz czas działania opisuje następująco zależność

\[   
T(P,D) \leq 
     \begin{cases}
       \alpha PD + T(P_{1},\ceil*{D/2}) + T(P_{2}, \floor{D/2}) &\quad\text{if } D > 1 \\
       \beta P &\quad\text{if } D \leq 1 \\
     \end{cases}
\]


gdzie $P_{1} + P_{2} \leq P$. Korzystając z zależności $\ceil*{D/2} \leq \frac{2D}{3}$ dla $D \geq 2$, możemy udowodnić że $T(P,D) \leq 3\alpha PD + \beta P$, co dowodzi że algorytm wykonuje się w $O((M+N)D)$ czasie.
\vspace{4mm}    

Natomiast nasza złożoność pamięciowa, zauważmy, że procedura poszukiwania środkowego ogona zajmuje $O(D)$ miejsca, ze względu na wykorzystanie dwóch wektorów V, dodatkowo procedura poszukiwania środkowego ogona działa niezależnie od rekurencji, więc na całą rekurencje potrzebujemy miejsca tylko na 2 wektory, natomiast rekurencja składa się z $O(logD)$ poziomów więc zajmuje $O(logD)$ pamięci, stąd mamy złożoność $O(m+n)$ pamięciową.
\end{document}
