\section{Dopasowanie wzorca z wieloznacznikami}

Wprowadźmy teraz symbol specjalny $?$ jako wieloznacznik lokalny (\emph{don't care symbol}), pasujący do każdego pojedynczego symbolu z $\A$.

\begin{definition}{}{}
  Relacja $\simeq$ jest relacją {\bf\textit{dopasowania z wieloznacznikiem lokalnym}} tj. $u \simeq v$ wtedy i tylko wtedy, gdy $|u| = |v|$ oraz dla wszystkich $1 \le i \le |u|$ zachodzi $u[i] = v[i]$ lub $u[i] =\,?$, lub też $v[i] =\,?$.
\end{definition}

\begin{algorithm}[H]
    \caption{Wyszukiwanie wszystkich wystąpień wzorca z wieloznacznikami lokalnymi w tekście}
    \Input{Słowa $t, w \in (\A \cup \{?\})^+$ ($|t| = n$, $|w| = m$)}
    \Output{Zbiór liczb $S = \{1 \le i \le n: t[i..(i + m - 1)] \simeq w\}$.}
\end{algorithm}

Istnieje bardzo pomysłowy algorytm obliczania wszystkich wystąpień wzorca z wieloznacznikami lokalnymi w tekście, oparty o szybką transformatę Fouriera.
Jak wiadomo, FFT umożliwia obliczanie dla dwóch wektorów $X$ i $Y$ o długości odpowiednio $n$ i $m$ (zakładając, że $n \ge m$) w czasie $O(n \log{n})$ operacji splotu tj. takiego $Z$, że dla wszystkich $0 \le i \le n - m$ zachodzi
\begin{align*}
  Z[i] = \sum_{j = 0}^{m - 1} X[i + j] Y[m - 1 - j].
\end{align*}
Zwróćmy uwagę, że -- wyjątkowo -- mamy w tym przypadku indeksowanie od $0$.

\begin{code}
\captionof{listing}{Obliczanie wszystkie wystąpień wzorca z wieloznacznikami lokalnymi w tekście}
\inputminted{python}{code/approximate-string-matching/basic-fft.py}
\label{alg:string-matching-dont-care-basic-fft}
\end{code}

\begin{theorem}{}{}
  Algorytm \ref{alg:string-matching-dont-care-basic-fft} wyznacza poprawnie wszystkie wystąpień wzorca z wieloznacznikami lokalnymi w tekście.
\end{theorem}

\begin{proof}
Kluczowa obserwacja jest następująca: wybierzmy dwie różne litery $a, b \in \A$ i wyznaczmy odpowiednio dla wszystkich $0 \le i \le n - 1$ oraz $0 \le j \le m - 1$
\begin{align*}
    X[i] = \text{$1$ gdy $t[i + 1] = a$}, & \text{w przeciwnym przypadku $X[i] = 0$,} \\
    Y[j] = \text{$1$ gdy $w[m - j] = b$}, & \text{w przeciwnym przypadku $Y[j] = 0$.}
\end{align*}
Wówczas
\begin{align*}
    Z[i] = \sum_{j = 0}^{m - 1} X[i + j] Y[m - 1 - j] = |\{0 \le j \le m - 1: t[i + j + 1] = a \land w[j + 1] = b \}|,
\end{align*}
a zatem $Z[i]$ jest po prostu zliczeniem wszystkich niedopasowań między tekstem $t[i + 1..i + m]$ a wzorcem $w$,
takich że w tekście występuje $a$, a we wzorcu $b$.
Powtarzając to dla wszystkich par liter w alfabecie możemy zliczyć wszystkie niedopasowania między $t[i + 1..i + m]$ a $w$ -- a zatem dopasowania będą na na tych pozycjach, na których niedopasowań brak.
\end{proof}

Jak łatwo zauważyć, algorytm \ref{alg:string-matching-dont-care-basic-fft} działa w czasie $O(n \log{n} |\A|^2)$.
