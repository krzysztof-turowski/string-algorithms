\documentclass[12pt]{article}
\usepackage{polski}
\usepackage[utf8]{inputenc}
\usepackage{geometry}
\usepackage{mathtools}
\usepackage{amsthm}
\usepackage{hyperref}
\newgeometry{left=1in,right=1in,top=1in,bottom=1in}

\newtheorem{theorem}{Twierdzenie}[section]
\newtheorem{lemma}{Lemat}[section]

\begin{document}

\begin{center}
    \Large{Algorytm 2.89-aproksymacyjny \\ dla problemu najkrótszego wspólnego nadsłowa} \\
    \large{autorstwa Shang-Hua Teng oraz Frances F. Yao \cite{Teng-Yao}}
\end{center}

\section{Notacja i definicje}

Dla danego na wejściu zbioru słów $S$ chcemy znaleźć najkrótsze słowo $\alpha$ będące nadsłowem tego zbioru, t.j. $\forall_{s \in S} \, s$ jest podsłowem $\alpha$.
Standardowo, zakładamy że dla żadnej pary słów nie zachodzi sytuacja, w której jedno jest podsłowem drugiego.
Wprowadzamy następujące definicje i oznaczenia:
\begin{itemize}
\item $|s|$ jest długością słowa $s$,
\item $|S|=\sum_{s \in S}\,|s|$,
\item $opt(S)$ jest długością wyniku optymalnego - najkrótszego nadsłowa słów ze zbioru $S$.
\end{itemize}
Dla każdej pary słów $s=uv$, $t=vw$:
\begin{itemize}
\item $ov(s,t)=|v|$ jest długością ``nakładania się'' na siebie słów $s$ i $t$,
\item $pref(s,t)=u$.
\end{itemize}
Dla takich słów $s$ i $t$ najkrótszym ich nadsłowem, w którym $s$ występuje przed $t$ jest $uvw=pref(s,t)t$.
Słowo takie nazywamy złączeniem słów $s$ i $t$ oraz oznaczamy je przez $\langle s,t \rangle$.
Łatwo zauważyć, że najkrótsze nadsłowo $\alpha$ jest równe $\langle s_{\pi(1)},s_{\pi(2)},\dots,s_{\pi(n)} \rangle$ dla pewnej permutacji $\pi$ zbioru $\{1,2,\dots,n\}$. \\

Problem znajdowania najkrótszego nadsłowa może zostać przeformułowany na problem grafowy.
Dla zbioru słów $S$ możemy stworzyć ważony graf skierowany $G=(S,E)$, w którym wagi krawędzi są równe $d(s,t)=|pref(s,t)|$.
Dla zbioru krawędzi zdefiniujmy $d(E)=\sum_{(s,t) \in E}\,d(s,t)$ i analogicznie $ov(E)=\sum_{(s,t) \in E}\,ov(s,t)$.
Prosta obserwacja pokazuje, że suma długości słów w cyklu $E$ jest równa $d(E)+ov(E)$. 

\section{Optymalne pokrycie cyklowe}

Pokryciem cyklowym grafu $G$ jest zbiór cykli taki, że każdy wierzchołek $G$ należy do któregoś z nich.
Optymalne pokrycie cyklowe jest pokryciem cyklowym z (w tym przypadku) najmniejszą możliwą sumą wag krawędzi.
Można je znaleźć między innymi za pomocą algorytmu węgierskiego \footnote{\url{https://en.wikipedia.org/wiki/Hungarian_algorithm}} działającego w czasie $O(n^3)$ . \\

Niech $C=\{c_1,c_2,\dots,c_k\}$ będzie pokryciem cyklowym.
By udowodnić współczynnik aproksymacji należy wprowadzić najpierw kilka lematów.

\begin{lemma}
$d(C) \leq opt(S)$ oraz $ov(C) \geq |S|-opt(S)$.
\end{lemma}
\begin{lemma}
Niech $ov_m=\sum_{C:|V(C)|=m}\,ov(C)$.
Niech $C'$ będzie podzbiorem $C$ powstałym przez usunięcie krawędzi z najmniejszym $ov(s,t)$ z każdego cyklu.
Wtedy $ov(C') \geq \sum_{i=2}^n ov_i/i$.
\end{lemma}
Warto zauważyć, że nadsłowo powstałe przez złączenie ze sobą wszystkich ścieżek z $C'$ uzyskuje $ov(C') \geq ov(C)/2$, a więc daje nam to tak samo dobrą gwarancję kompresji jak algorytm zachłanny.
\begin{lemma}[\cite{Blum-Jiang}]
Niech $c_1,c_2 \in C$ oraz $s_1 \in c_1, s_2 \in c_2$.
Wtedy $ov(s_1,s_2) \leq d(c_1)+d(c_2)$.
\end{lemma}
Dla słowa $s_i$ znajdującego się w cyklu $c \in C$ wprowadźmy oznaczenie $\langle s_i,c \rangle = \\ \langle s_i,s_{i+1},\dots,s_m,s_1,\dots,s_i \rangle$.
Stwórzmy zbiór $R$ wybierając dowolne słowo z każdego cyklu z $C$.
Niech $\alpha=\langle r_1,\dots,r_k \rangle$ będzie nadsłowem słów z $R$.
Słowo $\overline{\alpha}=\langle \langle r_1,c_1 \rangle,\dots,\langle r_k,c_k \rangle \rangle$ będziemy nazywać rozszerzeniem słowa $\alpha$ - jest ono nadsłowem wszystkich słów z $S$.
\begin{lemma}
$|\overline{\alpha}|=|\alpha|+d(C)$.
\end{lemma}

\subsection{Kanoniczne pokrycie cyklowe}

Dla cyklu $c = (s_1,s_2,\dots,s_m) \in C$ definiujemy $period(c)=pref(s_1,s_2)pref(s_2,s_3) \dots pref(s_m,s_1)$.
Mówimy, że słowo $s$ pasuje do cyklu $c$ jeśli $s$ jest podsłowem $period(c)^k$ dla pewnego $k$.
Oczywiście, jeśli $s$ należy do $c$, to $s$ pasuje do $c$. \\

$s$ może jednak pasować do innych cykli.
Niech $s$ należy do $c$ ale pasuje również do $c'$.
Możemy wtedy ``przenieść'' $s$ do $c'$ nie zmieniając sumy wag wszystkich krawędzi w obu cyklach.
Kanoniczne pokrycie cyklowe to takie pokrycie cyklowe, w którym każdy wierzchołek $s$ jest przyporządkowany cyklowi o najmniejszej wadze wśród wszystkich, do których $s$ pasuje.
W pokryciu takim zachodzi następująca własność:
\begin{lemma}
Niech $c_1,c_2 \in C$, gdzie $C$ jest kanoniczne oraz $s_1 \in c_1, s_2 \in c_2$.
Wtedy $ov(s_1,s_2)+ov(s_2,s_1) < max(|s_1|,|s_2|)+min(d(c_1),d(c_2))$.
\end{lemma}

\section{Greedy-Insert}

Niech $C_2$ będzie zbiorem cykli 2-elementowych.
Nadsłowo wszystkich słów należących do $C_2$ tworzymy używając algorytmu Greedy-Insert:
\begin{enumerate}
\item Podziel $C_2$ na zbiory $F$ i $G$. $F$ zawiera krótsze słowa z każdego cyklu, $G$ dłuższe.
\item $(f_1,f_2,\dots,f_n)$ - słowa z $F$ w dowolnej kolejności, \\
      $(g_1,g_2,\dots,g_n)$ - słowa z $G$ w kolejności w jakiej występują w nadsłowie $\eta$ otrzymanym standardowym algorytmem zachłannym.
\item $q_O=\langle f_1,g_1,g_2,f_2,f_3,g_3,g_4,f_4,\dots \rangle$ \\
      $q_E=\langle g_1,f_1,f_2,g_2,g_3,f_3,f_4,g_4,\dots \rangle$.
\item Zwróć słowo $q$ - krótsze ze słów $q_O,q_E$.
\end{enumerate}
Zbiór sąsiadujących ze sobą słów w $q_O$ i $q_E$ jest nadzbiorem sąsiadujących ze sobą słów w $C_2$ i $\eta$.
Wybierając lepsze ze słów $q_O,q_E$ otrzymujemy następujący lemat:
\begin{lemma}
$ov(q) \geq ov(C_2)/2 + ov(\eta)/2$.
\end{lemma}
Korzystając z własności algorytmu zachłannego ($ov(\eta) \geq (|G|-opt(G))/2$):
\begin{lemma}
$ov(q) \geq ov(C_2)/2 + (|G|-opt(G))/4$.
\end{lemma}

\section{Opis algorytmu}

Algorytm w całości prezentuje się następująco: \\

Wejście: $S=\{s_1,s_2,\dots,s_n\}$.
\begin{enumerate}
\item Znajdź optymalne pokrycie cyklowe $C$ w $S$ i przekształć je w pokrycie kanoniczne.
\item Z każdego cyklu wybierz dowolne słowo tworząc zbiór $R$.
\item Znajdź optymalne pokrycie cyklowe $CC$ dla zbioru $R$.
\item Uruchom Greedy-Insert na wszystkich cyklach długości 2, otrzymując słowo $q$.
\item We wszystkich pozostałych cyklach usuń krawędź z najmniejszym nakładaniem się słów i połącz wszystkie ścieżki w słowo $\alpha$.
\item Zwróć $\overline{\alpha}$ - rozszerzenie słowa $\alpha$.
\end{enumerate}

\section{Analiza algorytmu}

Wprowadźmy oznaczenie $d_2,d_3,d_4$ na sumę wag krawędzi dla wszystkich cykli długości odpowiednio $2,3$ oraz 4 lub większych.
Analogicznie $ov_2,ov_3,ov_4$ na $\sum\,ov(C)$ dla cykli długości $2,3$ oraz 4 lub większych.

Z lematu 2.5, sumując po wszystkich cyklach:
\begin{lemma}
$ov_2 \leq |G| + d_2/2$.
\end{lemma}
\begin{theorem}
$|\alpha| \leq (1+8/9)opt(S)$.
\end{theorem}
\begin{proof}
Z lematu 2.2 i 3.2:
\[ov(\alpha) \geq ov_2/2+(|G|-opt(G))/4+2ov_3/3+3ov_4/4\]
Z lematu 2.3 $ov_i \leq 2d_i$. Z lematu 2.1 wiemy też, że $|R| \leq opt(R)+ov(CC) \leq opt(S)+ov(CC)$.
\begin{align*}
    |\alpha| & = |R|-ov(\alpha) \\
        & \leq opt(S)+ov(CC)-ov_2/2-2ov_3/3-3ov_4/4 \\
        & = opt(S)+ov_2/2+ov_3/3+ov_4/4 \\
        & \leq opt(S)+d_2+(2/3)d_3+d_4/2
\end{align*}
Możemy też inaczej ograniczyć od góry $|\alpha|$. Korzystamy m.in. z lematu 5.1.
\begin{align*}
    |\alpha| & = |R|-ov(\alpha) \\
        & \leq |R|-ov_2/2-(|G|-opt(G))/4-2ov_3/3-3ov_4/4 \\
        & \leq |R|-ov_2/2-(ov_2-d_2/2-opt(S))/4-2ov_3/3-3ov_4/4 \\
        & = |R|-3ov_2/4-2ov_3/3-3ov_4/4+d_2/8+opt(S)/4 \\
        & = 5opt(S)/4+ov_2/4+ov_3/3+ov_4/4+d_2/8 \\
        & = 5opt(S)/4+ov(CC)/4+ov_3/12+d_2/8 \\
        & = (2-1/4)opt(S)+d_3/6+d_2/8
\end{align*}
Analizę algorytmu kończymy rozważając dwa przypadki:
\begin{enumerate}
\item $d_2 \leq (2/3)d(C)$, wtedy z pierwszej nierówności:
    \begin{align*}
        |\alpha| & \leq opt(S)+(2/3)d(C)+(2/3)(1/3)d(C) \\
            & \leq (1+8/9)opt(S)
    \end{align*}
\item $d_2 > (2/3)d(C)$, wtedy $d_3 \leq (1/3)d(C)$ i z drugiej nierówności:
    \begin{align*}
        |\alpha| & \leq (2-1/4)opt(S)+d_3/6+d_2/8 \\
            & = (2-1/4)opt(S)+(d_2+d_3)/8+d_3/24 \\
            & \leq (2-1/4)opt(S)+d(C)/8+d(C)/72 \\
            & \leq (1+8/9)opt(S)
    \end{align*}
\end{enumerate}
\end{proof}
Z lematu 2.4. $|\overline{\alpha}| = |\alpha|+d(C) \leq (2+8/9)opt(S)$, co kończy dowód współczynnika aproksymacji.

\begin{thebibliography}{3}
    \bibitem{Teng-Yao}
    S.-H. Teng, F.F. Yao.
    \textit{Approximating shortest superstrings.}
    SIAM J. Comput., 1997, Volume 26(2), pp. 410–417.

    \bibitem{Blum-Jiang}
    A. Blum, T. Jiang, M. Li, J. Tromp, M. Yannakakis.
    \textit{Linear approximation of shortest superstrings.}
    Proc. 23rd ACM Symposium on the Theory of Computing, ACM, New York, 1991, pp. 328–336.
\end{thebibliography}

\end{document}