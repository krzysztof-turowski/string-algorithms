\documentclass[
12pt
]{article}
\usepackage[T1]{fontenc}
\usepackage[utf8]{inputenc}
\usepackage{amsthm}
\usepackage{amsmath} 
\usepackage{hyperref}
\usepackage{cleveref}
\usepackage{sectsty}
\usepackage{algorithmicx,algpseudocode}
\usepackage{listings}

\MakeRobust{\Call}

\sectionfont{\fontsize{15}{15}\selectfont}

\newtheorem{theorem}{Theorem}
\newtheorem{corollary}{Corollary}[theorem]
\newtheorem{lemma}[theorem]{Lemma}
\newtheorem{obs}{Obserwacja}

\theoremstyle{definition}
\newtheorem{definition}{Definition}

\theoremstyle{remark}
\newtheorem*{remark}{Remark}


\algnewcommand\algorithmicinput{\textbf{INPUT:}}
\algnewcommand\Input{\item[\algorithmicinput]}

\algnewcommand\algorithmicoutput{\textbf{OUTPUT:}}
\algnewcommand\Output{\item[\algorithmicoutput]}

\begin{document}

\begin{center}
\Huge Algorytm szukania wzorca w tekście z $k$ błędami
\end{center}
\begin{center}
    na podstawie pracy \textit{Efficient string matching with k mismatches} napisanej przez  Gad M. Landau oraz Uzi Vishkin
\\
Opis przygotowany przez Pawła Madera
\end{center}


\section*{Wprowadzenie}
W podanym tekście $T[1,\dots,n]$ chcemy znaleźć wszystkie wystąpienia wzorca $A[1,\dots,m]$, które różnią się z nim na nie więcej niż $k$ pozycjach, w złożoności $O(k(m\ log\ m \ + \ n))$

\section*{Główna idea}
Niech PAT-MISMATCH będzie dwuwymiarową tablicą o rozmiarze $(m-1)(2k+1)$, gdzie 
$i$-ty wiersz tablicy zawiera $2k+1$ pierwszych pozycji, idąc od lewej strony, na których podsłowa $A[i+1:m]$ oraz $A[1:m-i]$ się różnią (PAT-MISMATCH$[i][j] = f$ oznacza, że $A[i+f] \neq A[f]$ oraz, że jest to niedopasowanie numer $j$). 
Domyślną wartością w tej tablicy jest $m+1$.

Niech TEXT-MISMATCH będzie dwuwymiarową tablicą o rozmiarze $(n-m+1)(k+1)$, w której $i$-ty wiersz zawiera informacje na temat $k+1$ pierwszych pozycji, idąc od lewej strony, na których podsłowo $T[i:i+m]$ różni się od wzorca $A$ (TEXT-MISMATCH$[i][j] = f$ oznacza, że $T[i+f] \neq A[f]$ oraz, że jest to niedopasowanie numer $j$). 
Domyślną wartością w tej tablicy jest $m+1$.

Dzięki tablicy TEXT-MISMATCH jesteśmy w stanie sprawdzić, czy podsłowo $T[i+1:i+m]$ jest dopasowaniem wzorca $A$ z $k$ niedopasowaniami.
Odpowiedź jest twierdząca, wtedy i tylko wtedy gdy TEXT-MISMATCH$[i][k+1] = m+1$. 

Pokażemy teraz jak wykorzystać tablicę PAT-MISMATCH do wyliczania wartości w tablicy TEXT-MISMATCH.
Niech $j$ będzie najdalszą pozycją niedopasowania litery wzorca, które znaleźliśmy w tekście.
Niech $r$ indeksem początku podsłowa tekstu, dla którego pozycja $j$ jest niedopasowaniem, tzn. $T[j] \neq A[j-r]$.
Obliczając wartości dla wiersza i-tego w tablicy TEXT-MISMATCH, takiego że $r\leq i < j$ możemy skorzystać z następującej obserwacji:
\begin{obs}
\label{obs_1}
Pozycje na których T[i+1:j] będzie się różniło z A[1:j-i] są pozycjami rozróżniającymi podsłowa A[i-r+1:j-r] i A[1:j-i] (przypadek 1) lub T[i+1:j] i A[i-r+1:j-r] (przypadek 2). 
\end{obs}

Dowód: Niech x będzie pozycją w tekście oraz $i+1\leq x < j$. Rozważmy następujące przypadki:
\begin{enumerate}
    \item A[x-r] = A[x-i] oraz T[x] = A[x-r], wtedy T[x] = A[x-i],
    \item A[x-r] = A[x-i] oraz $T[x] \neq A[x-r]$, wtedy $T[x] \neq A[x-i]$, 
    \item $A[x-r] \neq A[x-i]$ oraz T[x] = A[x-r], wtedy
    $T[x] \neq A[x-i]$,
    \item $A[x-r] \neq A[x-i]$ oraz $T[x] \neq A[x-r]$, wtedy może zajść $T[x] \neq A[x-i]$ i T[x] = A[x-i].
\end{enumerate}
Tak, więc jedynymi pozycjami na których T[i+1:j] będzie się różniło z A[1:j-i], są te na których różnią się podsłowa A[i-r+1:j-r] i A[1:j-i] lub T[i+1:j] i A[i-r+1:j-r].
\qed

Zauważmy, że poszukiwane pozycje są obliczone w tablicach PAT-MISMATCH[i-r] oraz TEXT-MISMATCH[r].

Niech operacja MERGE poszukuje pozycji rozróżniających słowa T[i+1:j] oraz A[1:j-i] przeglądając kolejne komórki w tablicach PAT-MISMATCH[i-r] oraz TEXT-MISMATCH[r], reprezentujące pozycje nie wykraczające poza j.
\begin{lemma}
\label{lem:pozycje}
Jeżeli jest $\geq k+1$ niedopasowań w pozycjach $\leq j$, wtedy MERGE znajduje pierwsze k+1 z nich. Jeżeli niedopasowań jest <k+1 w pozycjach $\leq j$, wtedy MERGE znajdzie je wszystkie. 
\end{lemma}

Dowód: Zauważmy, że pozycji spełniających przypadek 2 jest nie więcej niż k. Niech y będzie liczbą pozycji spełniających przypadek 1. Wykażemy, że nie potrzebujemy więcej niż 2k+1 pozycji spełniających przypadek 1, aby procedura MERGE działa poprawnie. Jeżeli:
\begin{enumerate}
\item PAT-MISMATCH[i-r][2k+1] >= j-i, to korzystając z obserwacji \ref{obs_1} znajdziemy wszystkie lub k+1 pierwszych niedopasowań. 
\item PAT-MISMATCH[i-r][2k+1] < j-i, tzn. mamy 2k+1 pozycji >i oraz <j, które rozróżniają podsłowa wzorca. 
Przypomnijmy, że pozycji spełniających warunek 1 jest nie więcej niż k. Wtedy co najmniej k+1 pozycji spełnia przypadek 1, a nie spełnia przypadeku 2.
Korzystając z obserwacji \ref{obs_1} procedura MERGE znajdzie k+1 pozycji rozróżniających słowa T[i+1:j] oraz A[1:j-i].
\end{enumerate}
\qed
\section{Algorytm analiza tekstu}
Niech procedura EXTEND(i, j, b) iteracyjnie przegląda słowa T[j:] oraz A[j-i:] dopóki nie znajdzie k+1 niedopasowań. Liczba b to ilość niedopasowań znaleziona przez procedurę MERGE.

Pokażemy teraz algorytm wyznaczający wartości tablicy TEXT-MISMATCH, bazując na tablicy PAT-MISMATCH (której wyliczenie zaprezentujemy w następnej sekcji).
\begin{lstlisting}
TEXT-MISMATCH[0,...,n-m][1,...,k+1] = m+1
r, j = 0, 0
for i in range(0, n-m):
    b = 0
    if i < j:
        MERGE(i, r, j, b)
    if b < k+1:
        r = i
        EXTEND(i, j, b)

\end{lstlisting}
\section{Analiza wzorca}
W tej sekcji zaprezentujemy obliczanie tablicy PAT-MISMATCH. 
Załóżmy, bez strat ogólności, że m jest potęgą dwójki. 
Podzielmy zbiór wierszy tablicy PAT-MISMATCH na log m następujących zbiorów: [1], [2,3], [4,5,6,7], ... [$\frac{1}{2}$m,...,m].

W kolejnych krokach analizy wzorca, będziemy wyliczali tablicę PAT-MISMATCH dla kolejnych zbiorów wierszy.
Analiza wzorca dla l-tego zbioru wierszy, tzn. dla wierszy [$2^{l-1}$, ..., $2^l-1$] przebiega w taki sam sposób jak analiza tekstu, przy czym rolę tablicy PAT-MISMATCH spełnia tablica PAT-MISMATCH[1:$2^{l-1}-1$], rolę tablicy TEXT-MISMATCH spełnia PAT-MISMATCH[$2^{l-1}$, ..., $2^l-1$], a $k$ zastępujemy przez min($2^{log\ m\ -\ l}2k+1, m-2^l$) (dowód, że takie k jest odpowiednie, jest analogiczny jak \ref{lem:pozycje})

\section*{Złożoność}
Każda faza analizy wzorca wykonywana jest w złożoności O($km$). 
Ponieważ mamy logm faz analizy wzorca dostajemy sumaryczną złożoność O$(km \ log \ m)$.
Analizy tekstu wykonana jest w złożoności O($kn$).
Złożoność całego algorytmu to $O(k(m\ log\ m \ + \ n))$.


\end{document}