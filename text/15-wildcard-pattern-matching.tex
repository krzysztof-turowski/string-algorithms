\section{Dopasowanie wzorca z wieloznacznikami}

Wprowadźmy teraz symbol specjalny $?$ jako wieloznacznik lokalny (\emph{don't care symbol}), pasujący do każdego pojedynczego symbolu z $\A$.

\begin{definition}{}{}
  Relacja $\simeq$ jest relacją {\bf\textit{dopasowania z wieloznacznikiem lokalnym}} tj. $u \simeq v$ wtedy i tylko wtedy, gdy $|u| = |v|$ oraz dla wszystkich $1 \le i \le |u|$ zachodzi $u[i] = v[i]$ lub $u[i] =\,?$, lub też $v[i] =\,?$.
\end{definition}

\begin{algorithm}[H]
    \caption{Wyszukiwanie wszystkich wystąpień wzorca z wieloznacznikami lokalnymi w tekście}
    \Input{Słowa $t, w \in (\A \cup \{?\})^+$ ($|t| = n$, $|w| = m$)}
    \Output{Zbiór liczb $S = \{1 \le i \le n: t[i..(i + m - 1)] \simeq w\}$.}
\end{algorithm}

Istnieje bardzo pomysłowy algorytm obliczania wszystkich wystąpień wzorca z wieloznacznikami lokalnymi w tekście, oparty o szybką transformatę Fouriera.
Jak wiadomo, FFT umożliwia obliczanie dla dwóch wektorów $X$ i $Y$ o długości odpowiednio $n$ i $m$ (zakładając, że $n \ge m$) w czasie $O(n \log{n})$ operacji splotu tj. takiego $Z$, że dla wszystkich $0 \le i \le n - m$ zachodzi
\begin{align*}
  Z[i] = \sum_{j = 0}^{m - 1} X[i + j] Y[m - 1 - j].
\end{align*}
Zwróćmy uwagę, że -- wyjątkowo -- mamy w tym przypadku indeksowanie od $0$.

\begin{code}
\captionof{listing}{Obliczanie wszystkie wystąpień wzorca z wieloznacznikami lokalnymi w tekście}
\inputminted{python}{code/approximate-string-matching/basic-fft.py}
\label{alg:string-matching-dont-care-basic-fft}
\end{code}

\begin{theorem}{}{}
  Algorytm \ref{alg:string-matching-dont-care-basic-fft} wyznacza poprawnie wszystkie wystąpień wzorca z wieloznacznikami lokalnymi w tekście.
\end{theorem}

\begin{proof}
Kluczowa obserwacja jest następująca: wybierzmy dwie różne litery $a, b \in \A$ i wyznaczmy odpowiednio dla wszystkich $0 \le i \le n - 1$ oraz $0 \le j \le m - 1$
\begin{align*}
    X[i] = \text{$1$ gdy $t[i + 1] = a$}, & \text{w przeciwnym przypadku $X[i] = 0$,} \\
    Y[j] = \text{$1$ gdy $w[m - j] = b$}, & \text{w przeciwnym przypadku $Y[j] = 0$.}
\end{align*}
Wówczas
\begin{align*}
    Z[i] = \sum_{j = 0}^{m - 1} X[i + j] Y[m - 1 - j] = |\{0 \le j \le m - 1: t[i + j + 1] = a \land w[j + 1] = b \}|,
\end{align*}
a zatem $Z[i]$ jest po prostu zliczeniem wszystkich niedopasowań między tekstem $t[i + 1..i + m]$ a wzorcem $w$,
takich że w tekście występuje $a$, a we wzorcu $b$.
Powtarzając to dla wszystkich par liter w alfabecie możemy zliczyć wszystkie niedopasowania między $t[i + 1..i + m]$ a $w$ -- a zatem dopasowania będą na na tych pozycjach, na których niedopasowań brak.
\end{proof}

Jak łatwo zauważyć, algorytm \ref{alg:string-matching-dont-care-basic-fft} działa w czasie $O(n \log{n} |\A|^2)$.

\subsection{Algorytm Clifforda-Clifforda}

Na wejściu do algorytmu dostajemy tekst i wzorzec w których oprócz liter mogą występować wildcardy (?) pasujące do dowolnego znaku. Na wyjściu ma się znaleźć lista pozycji w których wzorzec pasuje do tekstu.

\paragraph{Wersja bez wildcardów}

W rozwiązaniu problemu użyty jest splot obliczany za pomocą szybkiej transformaty Fouriera (FFT) zdefiniowany następująco:

$$
    p \otimes t \defeq (\sum^{m-1}_{j=0}p_{j}t_{i+j}, 0 \leq i \leq n-m)
$$

Normalne (bez wildcardów) dopasowanie wzorca p do tekstu t na pozycjach od i do i+m możemy obliczyć za pomocą splotu. Korzystamy z własności wzoru:

$$
    \sum_{j=0}^{m-1} (p_j - t_{i+j})^2 = \sum_{j=0}^{m-1} (p^2_{j} - 2 p_{j}t_{i+j}+t^2_{i+j})
$$

Zobaczmy że lewa część równania równa 0 to znaleźliśmy dopasowanie. Tekst nie różni się od wzorca na żadnej z m kolejnych pozycji. Prawą stronę potrafimy szybko obliczyć ponieważ podniesienie każdej pozycji do kwadratu wykonywane jest w czasie stałym a środkowy składnik liczmy za pomocą FFT.

\paragraph{Wersja z wildcardami}

Zdefiniujmy ciągi:
\begin{align*}
p'_j =
    \begin{cases}
    0  &  p_j = '?'\\
    1  &  wpp
    \end{cases}
\end{align*}
\begin{align*}
t'_j =
    \begin{cases}
    0  &  t_j = '?'\\
    1  &  wpp
    \end{cases}
\end{align*}

Wtedy łatwo jest widać że następujące równanie jest naturalnym rozszerzeniem rozwiązania poprzedniego problemu.

    $$ \sum_{j=0}^{m-1} p'_{j}t'_{i+j}(p_j - t_{i+j})^2 = 0 $$

Zobaczmy że tekst część pod sumą jest zawsze nieujemna $p'_{j}$ oraz $t'_{i+j}$ przyjmują wartości \{0,1\} oraz kwadrat różnicy jest zawsze nieujemny. Więc zastanówmy się kiedy suma jest równa zero. Wtedy kiedy wszystkie składniki są równe zero. Kiedy składnik jest równy zero? W jednym z 3 przypadków. $p'_{j} = 0$ wtedy mamy wildcard we wzorcu, $t'_{i+j}=0$ wtedy mamy wildcard w tekście, $(p_j - t_{i+j})^2$ wtedy litery wzorca i tekstu są równe. Zobaczmy że to zachodzi wtedy i tylko wtedy gdy mamy do czynienia z dopasowaniem.

Po rozpisaniu nawiasu dostajemy formę:

$$ \sum_{j=0}^{m-1} (p'_{j}p^2_{j}t'_{i+j} - 2 p'_{j}p_{j}t_{i+j}t'_{i+j}+p'_{j}t^2_{i+j}t'_{i+j}) $$

Takie rozwinięcie łatwo jest obliczyć wykorzystując 3 razy FFT oraz podstawowe operacje na ciągach.

\paragraph{Złożoność}

Trywialnie jest pokazać że algorytm działa $O(n \log n)$ gdy użyjemy FFT dla tekstu i patternu rozszerzonego do długości tekstu. Można jednak osiągnąć $O(n \log m)$ gdy podzielimy tekst na $n/m$ zachodzących na siebie kawałków wielkości $2m$ i dla każdego policzymy matching ze wzorcem a następnie wyniki złączymy. Zobaczmy że jest to poprawne ponieważ dla każdego miejsca początkowego istnieje kawałek, który zawiera część od długości co najmniej $m$ od tego miejsca. Złożoność: $n/m * O(m\log m)=O(n\log m)$.
