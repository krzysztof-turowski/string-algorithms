\section{Algorytm aproksymacyjnego dopasowania wzorca względem odległości edycyjnej - podejście na podstawie algorytmu Boyer'a-Moore'a.}

\begin{algorithm}[H]
    \caption{$k$ - przybliżone wyszukiwanie wzorca w tekście względem odległości edycyjnej}
    \Input{Słowa $t, p \in \A^+$ ($|t| = n$, $|p| = m$)}
    \Output{Zbiór liczb $S = \{j : \text{ istnieje podsłowo T zakończone na $t_j$ o odległości edycyjnej od p równej co najwyżej $k$}\}$.}
\end{algorithm}

\subsection{Algorytm programowania dynamicznego.}
Podstawowym algorytmem rozwiązującym zadany wyżej problem jest następujący algorytm programowania dynamicznego wyliczający tablicę $D(i,j)$ oznaczającą minimalną odległość edycyjną między słowami $p_1...p_i$ oraz dowolnym podsłowem $T$ zakończonym na $t_j$:
\begin{align*}
    & D(0,j) = 0,\ 0 \leq j \leq n \\
    & D(i, j) = min\ \left\{\begin{array}{ll}
         &  D(i-1,j)+1\\
         &  D(i-1,j-1) + \text{ if } p_i = t_j \text{ then } 0 \text{ else } 1 \\
         &  D(i, j-1)+1
    \end{array} \right.
\end{align*}
Rozwiązaniem są wszystkie $j$ takie, że $D(m,j) \leq k$. Algorytm ten działa w czasie $O(mn)$.

\subsection{Idea algorytmu ABM (Approximate Boyer-Moore)}

Algorytm oparty na pomyśle analogicznym do algorytmu Boyer'a-Moore'a będzie będzie działać w dwóch głównych fazach: \textit{skanowania} i \textit{sprawdzania}.
W fazie skanowania przechodzimy przez dane słowo $t$ i zaznaczamy pewne komórki pierwszego rzędu tablicy $D$ (tj komórki postaci $D(0,j)$).
Po zakończeniu fazy skanowania rozpoczynamy fazę sprawdzania, czyli dla każdej zaznaczonej komórki $D$ będziemy obliczali przekątną tablicy $D$ poprzez zwykłe programowanie dynamiczne.
Kiedykolwiek nasza procedura dynamiczna będzie odnosić się do niewyliczonej komórki możemy przyjąć, że jej wartość wynosi $\infty.$\\
Faza sprawdzania wydaje się dość intuicyjna dlatego skupmy się na fazie skanowania.\\

Faza skanowania powtarza w pętli dwie czynności. \textit{Zaznacza} (mark) i \textit{przesuwa} (shift) indeks który skanujemy. Operacja przesuwania jest analogiczna do operacji przesuwania w algorytmie $Boyer'a-Moore'a$. Operacja zaznaczania wykonywana jest wtedy gdy obecny indeks wymaga dokładniejszego sprawdzenia przez wyliczenie tablicy $D.$

\subsection{Faza skanowania algorytmu ABM}
\begin{definition}{}{}
    Mówimy, że dla każdego $D(i,j)$ istnieje {\bf \textit{łuk minimalizujący}} z $D(i-1, j)$ do $D(i, j)$ jeśli $D(i,j) = D(i - 1, j)+1$, z $D(i, j-1)$ do $D(i, j)$ jeśli $D(i,j) = D(i, j-1)+1$, oraz z $D(i-1,j-1)$ do $D(i, j)$ jeśli $D(i,j) = D(i-1, j-1)$ gdy $p_i = t_j$ albo
    $D(i,j) = D(i-1, j-1) + 1$ gdy $p_i \neq t_j.$
\end{definition}
\begin{definition}{}{}
    Dla każdego $1 \le h \le n$, {\bf \textit{minimalizującą ścieżką}} nazywamy dowolną ścieżkę złożoną z minimalizujących łuków zaczynającą w $D(0,j)$ w pierwszym rzędzie tablicy $D$ do komórki $D(m, h)$ znajdującej się w ostatnim rzędzie. Minimalizującą ścieżkę nazywamy {\bf \textit{dobrą}} gdy prowadzi do $D(m, h) \leq k.$
\end{definition}
\begin{lemma}{}{}
    Komórki dowolnej dobrej ścieżki minimalizującej zawarte są wśród $\leq k+1$ kolejnych przekątnych tablicy $D.$
\end{lemma}
\begin{definition}{}{}
    Dla $i = 1,...m$ przez {\bf \textit{$k$-otoczenie}} znaku $p_i$ ze wzorca $p$ oznaczmy słowo $C = p_{i-k}...p_{i+k}$, gdzie $p_j = \epsilon$ dla $j < 1$ oraz $j > m.$
\end{definition}
\begin{lemma}{}{}
    \label{min-diag}
   Jeśli dobra ścieżka minimalizująca przechodzi przez pewne komórki przekątnej $h$ tablicy $D$, wtedy dla co najwyżej $k$ indeksów $i$, $1 \leq i \leq m$, znak $t_{h+i}$ nie występuje w $k$-otoczeniu $C_i.$
\end{lemma}

\begin{definition}{}{}
    Kolumnę $j$, $h+1 \leq j \leq h+m$ tablicy $D$ nazywamy {\bf \textit{złą}} jeśli $t_j$ nie należy do $k$-otoczenia $C_{j-h}.$
\end{definition}

\begin{remark-thm}
    Z lematu $\ref{min-diag}$ wynika, że dla danego $t_j$ złych kolumn jest co najwyżej $k$ o ile
\end{remark-thm}

Na podstawie powyższej obserwacji i lematu możemy skonstruować następującą procedurę zaznaczania. Dla przekątnej $h$, dla $i=m, m-1, ..., k+1$ lub dopóki nie znaleźliśmy $k+1$ złych kolumn sprawdź czy $t_{h+1}$ należy do $C_i$.
Jeśli znaleźliśmy $\leq k$ złych kolumn wtedy zaznaczmy komórki $D(0, h-k),...,D(0,h+k)$.\\
Aby szybko odpowiadać na to czy kolumna jest zła możemy obliczyć tablicę $BAD(i,a)$ dla $1 \leq i \leq m$, $a \in \A$ taką, że
\[
    Bad(i,a) = \text{ \textbf{ true}, wtw  gdy $a$ \textbf{nie} należy do $k$-otoczenia $C_i$.}
\]
Taką tablicę dla wzorca $p$ możemy obliczyć w czasie $O((|\A|+k)m)$.\\
Po zbadaniu przekątnej $h$ możemy ustalić przesunięcie $d$ tak aby rozpocząć sprawdzanie od przekątnej $d$. Oczywistym jest, że minimalnie $d$ może wynosić $k+1$ i niczego nie pominiemy, jednak oby zrobić to lepiej, możemy skorzystać z podejścia analogicznego do algorytmu $Boyer'a-Moore'a$. Skorzystajmy tutaj z tablicy przesunięć $BM_k[i,a] = min \{s:\ s=m \text{ lub } (1\leq s < m oraz p_{i-s} = a)\}$ wymiaru $(k+1) \times |\A|$. Poniższy algorytm oblicza ją w czasie $O(m+k|\A|)$:

\begin{code}
\captionof{listing}{Obliczanie tablicy przesunięć Boyer'a-Moore'a}
\inputminted{python}{bm_multidim.py}
\label{alg:exact-string-matching-morris-pratt}
\end{code}

Cała faza skanowania opisana jest w następującym kodzie:

\begin{code}
\captionof{listing}{Faza skanowania algorytmu ABM}
\inputminted{python}{scan_faze.py}
\label{alg:exact-string-matching-morris-pratt}
\end{code}

\subsection{Złożoność.}

Obliczanie tablic $BAD$ i $BM_k$ przy założeniu, że $k < m$ zajmuje $O((k+|\A|)m).$ Cały algorytm skanowania w najgorszym przypadku zajmuje $O(\frac{mn}{k}).$ Autorzy pracy \url{https://www.researchgate.net/publication/220616992_Approximate_Boyer-Moore_String_Matching} udowodnili (dość nietrywialnie) następujące twierdzenie:

\begin{theorem}{}{}
    Dla $2k+1 < |\A|$ oczekiwana czas trwania Algorytmu 2 zajmuje $O(\frac{|\A|}{|\A| - 2k} \cdot kn \cdot (\frac{k}{|\A| + 2k^2} + \frac{1}{m})).$
\end{theorem}

Fazę sprawdzania można zaimplementować tak by działała w czasie $O(kn),$ co sprawia, że cały algorytm $ABM$ rozwiązuje problem $k$-przybliżonego dopasowania względem odległości edycyjnej w oczekiwanym czasie $O(kn\cdot(\frac{1}{m-k} + \frac{k}{|\A|}))$
