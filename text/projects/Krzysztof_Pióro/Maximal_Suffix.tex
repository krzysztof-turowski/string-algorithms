\documentclass{article}
\usepackage{amsmath}
\usepackage{amsthm}
\usepackage{amssymb}
\usepackage{algorithm}
\usepackage[noend]{algpseudocode}
\usepackage{polski}
\usepackage[utf8]{inputenc}

\algrenewcommand\textproc{}
\renewcommand{\thealgorithm}{}


\title{Proste wyznaczenie maksymalnego sufiksu słowa\\ \large Na podstawie ''A note on a simple computation of the maximal suffix
of a string'' -- Adamczyk, Rytter}
\author{Krzysztof Pióro}
\date{Maj 2022}

\begin{document}

\maketitle

\section{Wstęp}

Autorzy pracy pokazują prosty w opisie algorytm wyznaczający maksymalny sufiks słowa działający w czasie liniowym i stałej dodatkowej pamięci.

\section{Definicje}

Będziemy rozważać słowo $w$ długości $n$ indeksowane od $1$ do $n$. Dodatkowo definiujemy:

\begin{itemize}
    \item MaxSuf$(w)$ -- maksymalny sufiks słowa $w$
    \item period$(w)$ -- najkrótszy okres słowa $w$
    \item słowo nazywamy \textit{samo-maksymalnym} $\iff$ MaxSuf$(w) = w$.
    \item MaxSufPos$(w)$ -- pozycja od której zaczyna się maksymalny sufiks słowa $w$
    \item $w_1 < w_2$ -- $w_1$ jest mniejsze leksykograficznie od $w_2$
\end{itemize}



\section{Algorytm}

\begin{algorithm}[H]
    \caption{\textbf{Compute-MaxSufPos$(w)$}}
    \begin{algorithmic}
        \State $i := 1$; $j := 2$;
        \While {$j \leq n$}
            \State $k := 0$
            \While {$j + k < n$ \textbf{and} $w[i+k] = w[j+k]$}
                \State $k := k + 1$;
            \EndWhile
            \If{$w[i+k] < w[j+k]$}
                \State $i := i + k + 1$;
            \Else
                \State $j := j + k + 1$;
            \EndIf
            \If {$i = j}$
                \State $j := j + 1$;
            \EndIf
        \EndWhile
        \State \Return $i$
    \end{algorithmic}
\end{algorithm}


Algorytm oczywiście działa w stałej (dodatkowej) pamięci i w liniowej liczbie operacji.

\section{Dowód poprawności}

Niech $(i, j) \rightarrow (i', j')$ oznacza, że z konfiguracji $(i, j)$  w jednej iteracji przechodzimy do $(i', j')$ i niech $\rightarrow^{*}$ będzie
domknięciem przechodnim relacji $\rightarrow$. 

Pokażemy, że po każdej głównej iteracji będą zachodziły poniższe \textbf{niezmienniki} (gdzie oznaczamy $u := w[i\ldots j-1]$):
\begin{enumerate}
    \item $(i < j < n) \Rightarrow u$ jest \textit{samo-maksymalne} oraz $period(u) = u$
    \item maksymalny sufiks słowa $w$ nie zaczyna się przed $i$
\end{enumerate}



\begin{proof}[\textbf{Dowód niezmienników}]
Początkowo $i = 1, j = 2$ i niezmiennik jest spełniony

Rozważmy iterację, w której $i$ jest zmienione po raz pierwszy. Pokażemy, że niezmienniki są spełnione przed tą iteracją 
(autorzy pracy piszą tutaj tylko, że jest to "easy to see", ale my to uzasadnimy). Oczywiście drugi niezmiennik jest spełniony, zatem skupimy się na pierwszym.

Rozważamy zatem przejście $(i=1, j) \rightarrow (i=1, j' = j + k + 1)$, gdzie niezmienniki zachodzą dla słowa $u = w[i \ldots j-1]$.

Wtedy niech $u' := w[i\ldots j + k] = u^tva$, gdzie $v$ jest ścisłym prefiksem $u$ (być może pustym), a $a$ jest literką taką, że $va < u$.

Z \textit{samo-maksymalności} słowa $u$ oraz faktu, że $va < u$ możemy wywnioskować, że słowo $u'$ również jest \textit{samo-maksymalne}.
Podobnie możemy też wywnioskować, że słowo $u^tv$ jest \textit{samo-maksymalne} (skorzystamy z tego później). 

Aby udowodnić, że $period(u') = u'$ pokażemy, że $u'$ nie ma żadnego właściwego prefikso-sufiksu. 
Każdy prefikso-sufiks $u'$ musi być przedłużeniem prefikso-sufiksu $w[i\ldots j+k-1]$. 
Najdłuższy prefikso-sufiks słowa $w[i\ldots j+k-1]$ to prefiks $p = w[i\ldots i+k-1]$ (odpowiada mu sufiks $s = w[j\ldots j+k-1] = u^{t-1}v$).
Z działania algorytmu wynika, że po prefiksie $p$ występuje litera $b$ taka, że $b > a$. 
Zauważmy dodatkowo, że wszystkie literki występujące po prefikso-sufiksach słowa $p$ są większe bądź równe $b$ (wpp. mielibyśmy sprzeczność z \textit{samo-maksymalnością $u^tv$)}.
Zatem nie istnieje prefikso-sufiks słowa $w[i\ldots j+k-1]$, którego dałoby się rozszerzyć do prefikso-sufiksu słowa $u'$, czyli $period(u') = u'$.

\vspace{5pt}

\textbf{Pierwsza modyfikacja zmiennej i:} Rozpatrzmy zatem pierwszą zmianę zmiennej $i$ z $i=1$ na $i' = i + k + 1$. 

Wtedy $w[i\ldots j+k] = u^t vb$, gdzie $v$ jest ścisłym prefiksem $u$ oraz $u < vb$.\\
Niech $m$ będzie pierwszą pozycją za $u^t$. Pokażemy, że (częściowa) historia algorytmu wygląda jak poniżej:
\begin{align*}
    (i, j) \rightarrow (i', j) \rightarrow^{*} (m, j) \rightarrow^{*}(m, m+1)
\end{align*}

Z niezmienników dla słowa $u$ możemy wywnioskować, że $u$ jest najmniejszym okresem słowa $u^tv$ ($u$ oczywiście jest okresem, a gdyby istniał jakiś mniejszy, to byłby on też okresem $u$) 
oraz to, że słowo $u^tv$ jest \textit{samo-maksymalne}. 

Zauważmy, że jeśli zmienna $i$ zostanie przeniesiona na dowolną pozycję z zakresu $[i', m-1]$, to następna pozycja dla $i$ nie może być większa niż $m$.
Wynika to z tego, że w takim przypadku znalezlibyśmy prefikso-sufiks słowa $u$, co byłoby sprzeczne z
$period(u) = u$. Z \textit{samo-maksymalności} słowa $u$ możemy dodatkowo wywnioskować, że dla takiej zmiennej $i$
następna iteracja zwiększy tą zmienną $i$. Zatem wartość zmiennej $i$ będzie zwiększana z $i'$, aż osiągnie $m$.

Następnie zwiększana będzie wartość zmiennej $j$, aż do momentu, gdy osiągnie ona wartość $m+1$ (możemy to 
uzasadnić podobnymi argmentami jak dla zmiennej $i$). 

Prześliśmy więc z $(i, j)$ do $(m, m+1)$. Z faktu, że $u < vb$, wiemy, że maksymalny sufiks 
nie może zacząć się przed pozycją $m$. Zatem 
możemy odciąć prefiks $u^t$ słowa $w$ i rozpocząć całe obliczenia od nowa dla mniejszego słowa(traktujemy $m$ jako $1$).
Ostatecznie dowód poprawności niezmiennika otrzymujemy z poprawności niezmiennika dla mniejszych słów.
\end{proof}

\begin{proof}[\textbf{Dowód poprawności algorytmu}]
Rozważmy teraz ostatnią wartość zmiennej $i$ oraz przedostatnią wartość zmiennej $j$.
Zgodnie z niezmiennikiem otrzymujemy, że słowo $u = w[i\ldots j-1]$ jest \textit{samo-maksymalne}.
Dodatkowo sufiks $w[i\ldots n]$ jest postaci $u^tva$, gdzie $v$ jest prefiksem $u$, a $a$ jest taką literką, że $va \leq u$. 
Z tej postaci sufiksu możemy już łatwo wywnioskować, że $w[i\ldots n]$  jest maksymalnym sufiksem, co kończy dowód poprawności algorytmu.

Warto tutaj zwrócić uwagę, że autorzy błędnie stwierdzili, że $u$ jest okresem sufiksu $w[i\ldots n]$. Możemy mieć bowiem
taką ostatnią literkę $a$, że $va < u'$, gdzie $u'$ to prefiks $u$ długości $|va|$ (ostatnie porównanie w algorytmie zwraca, że ostatnia litera jest mniejsza od tej, z którą ją porównywaliśmy).
\end{proof}

\end{document}