\section{Simple algorithms for string-searching}

%Naive serach section
\subsection{Naive search}
Most natural approach to solve string-searching is naive search where for each index $i$ in text $t$ we compare a substring $t[i : i+m]$ with the pattern. This algorithm can be written as follows:
\begin{minted}[xleftmargin=20pt, linenos]{python}
# t - text, s - pattern, n - length of t, m - length of s
def naive_search(t, s, n, m):
    for i in range(0, n-m):
        equals_on_index = True
        for j in range(0,m):
            if t[i+j] != s[j]:
                equals_on_index = False
                break
        if equals_on_index:
            return True
    return False
\end{minted}
It is very simple algorithm, but in the worst case it can be very slow. For example $t = a^n$, $s = a^{m-1}b$ requires $(n-m) \cdot m$ operations. It is easy to observe that the worst time complexity will be $\bigO (n \cdot m)$. For version with multiple patterns we have to run this algorithm for each pattern independently, which results in $\bigO(n \cdot \sum m_i)$ running time. \newline

However, if the patterns are generated randomly and the size of the alphabet is greater than $1$, this algorithm is quite fast. We observe that with such assumptions the expected number of comparisons in inner loop will be constant and for version with $q$ patterns we obtain $\bigO(q \cdot n)$ running time. 
\newline
\begin{table}[!htb]
\begin{center}
\caption{Comparison of simple algorithms}
\label{table:simpleAlgos}
\begin{tabular}{|c|c|c|c|}
\hline
\rowcolor[HTML]{C0C0C0} 
Algorithm          & Average case & Worst case & Additional space \\ \hline
Naive algorithm    & $\bigO(n)$       & $\bigO(n\cdot m)$     & $\bigO(1)$           \\ \hline
Rabin-Karp \cite{rabinKarp}        & $\bigO(n)$       & $\bigO(n\cdot m)$     & $\bigO(1)$           \\ \hline
Knuth-Morris-Pratt \cite{KMP} & $\bigO(n)$       & $\bigO(n)$     & $\bigO(m)$           \\ \hline
Two-way algorithm \cite{twoWayAlgo}  & $\bigO(n)$         & $\bigO(n)$       & $\bigO(\log m)$       \\ \hline
\end{tabular}
\end{center}
\end{table}

If the average complexity of a naive algorithm is not enough in specific case there are some algorithms mentioned in table \ref{table:simpleAlgos}, which for single pattern achieves optimal running time.
