\documentclass[a4paper,11pt]{article}
\usepackage[utf8]{inputenc}
\usepackage{polski}
\usepackage[utf8]{inputenc}
\usepackage[polish]{babel}
\usepackage[]{amsthm} %lets us use \begin{proof}
\usepackage[]{amssymb} %gives us the character \varnothing
\usepackage{amsmath}
\usepackage{algorithm}
\usepackage{algorithmic}
\newcommand{\algorithmicinput}{\textbf{input}}
\newcommand{\INPUT}{\item[\algorithmicinput]}

\title{Simple string matching with k mismatches \\ \large W oparciu o ,,Simple and efficient string matching with k mismatches“\\ R. Grossi, F. Luccio}
\author{Maciej Nemś}
\date{Maj 2022}

\begin{document}

\maketitle

\section{Wstęp}
Problem \textit{string matching with k mismatches} (SMK) polega na tym, że dla tekstu $T$ oraz dla wzorca $W$ takich, że $|T| = n$, $|W| = m$ znajdujemy wszystkie wystąpienia $W$ w $T$ z co najwyżej $k < m$ różnymi symbolami. W pracy autorzy przedstawiają dwa algorytmy rozwiązujące ten problem. Oba opierają się na liczbie permutacji znaków $W$ w $T$ z co najwyżej $k$ błędami. Pierwszy algorytm działa w czasie $O(n\log|A_W|+rm)$, gdzie $r$ to liczba wystąpień permutacji $W$ w $T$ z $k$ błędami, natomiast $A_w$, to zbiór znaków w $W$. Drugi algorytm działa w czasie $O(n\log|A_W| +dk)$, gdzie $d \leq r$, to liczba różnych wystąpień permutacji $W$ w $T$ z $k$ błędami. Oba algorytmy najpierw szukają wszystkich miejsc, gdzie w $T$ jest jakaś permutacja $W$ z $k$ błędami, a następnie zwracają z tych miejsc wszystkie te, które mają co najwyżej $k$ błędów.

\section{Opis algorytmów}

\subsection{Szukanie permutacji}
\subsubsection*{Algorytm}
Oba algorytmy rozwiązujące problem SMK wymagają najpierw obliczenia wszystkich miejsc w tekście $T$, gdzie zaczyna się jakaś permutacja wzorca $W$ z co najwyżej $k$ błędami. Do szukania permutacji będziemy potrzebować kolejki FIFO $Q$, która będzie trzymać symbole z $T$ oraz struktury $C$, która będzie trzymała liczniki dla każdego znaku występującego w $A_T$. Licznik $Z$ będzie zliczał liczbę błędów.
\par
Algorytm dla każdego znaku w tekście najpierw go dorzuca do kolejki i dekrementuje licznik. Jeśli $C[znak] < 0$, to znaczy, że wykorzystaliśmy już wszystkie wystąpienia tego znaku w $W$, więc mamy błąd (inkrementujemy $K$). Teraz tak długo, aż błędów w kolejce jest ponad $k$, usuwamy znaki z kolejki i odpowiednio dekrementujemy $K$. Po pętli, jeśli rozmiar $Q$ równa się $m$, to zebraliśmy tyle znaków, co długość wzorca, mamy co najwyżej $K$ błędów, czyli wiemy, że to jest permutacja wzorca z co najwyżej $K$ błędami i możemy ją zgłosić jako match. Dodatkowo, aby zawsze na początku pętli rozmiar $Q$ wynosił co najwyżej $m-1$, to usuwamy pierwszy element.
\begin{algorithm}[H]
\caption{Permutation Matching}
\begin{algorithmic} 
\INPUT{$T$, $W$, $n$, $m$, $k$}
\IF{$n < m$}
    \STATE{end with no matches}
\ENDIF
\STATE{\textbf{Preprocessing:} For each $a \in A_W$ set $C[a]$ to count of $a$ in $W$; For all $t \in T$ if $t \notin A_W$, substitute it with special symbol $\%\notin A_W$ to get new word $\tilde{T}$; Set $C[\%] = 0$}
\STATE{$Z:=0$}
\FOR{$j=1$ \TO $n$}
    \STATE{$Q.push(\tilde{T}[j])$}
    \STATE{$C[\tilde{T}[j]] = C[\tilde{T}[j]] - 1$}
    \IF{$C[\tilde{T}j]] < 0$}
        \STATE{$Z = Z + 1$}        
    \ENDIF\\
    \STATE{// Pop from $Q$, until there are at most $k$
    mismatches}
    \WHILE{$Z > k$}
        \STATE{$x = Q.pop()$}
        \IF{$C[x] < 0$}
            \STATE{$Z = Z - 1$}
        \ENDIF
        \STATE{$C[X] = C[X] + 1$}
    \ENDWHILE
    \STATE{// If $Q$ has length $m$ report occurance, as there are at most $k$ mismatches}
    \STATE{// also prepare for next iteration by poping from $Q$}
    \IF{$Q.size() == m$}
        \STATE{Report occurence in position $j - m + 1$}
        \STATE{$x = Q.pop()$}
        \IF{$C[x] < 0$}
            \STATE{$Z = Z - 1$}
        \ENDIF
        \STATE{$C[X] = C[X] + 1$}
    \ENDIF
\ENDFOR
\end{algorithmic}
\end{algorithm}
\subsubsection*{Poprawność}
Dość łatwo zauważyć, że algorytm ten jest poprawny. W każdej iteracji na koniec sprawdzamy warunki konieczne, by pozycja $j-m+1$ była dopasowaniem. Tzn sprawdzamy liczbę zebranych znaków w kolejce i liczbę błędów. Przesuwamy się w każdym wykonaniu pętli o $1$, więc sprawdzimy wszystkie możliwe pozycje.
\subsubsection*{Złożoność}
Złożoność takiego dopasowania to $O(n\log |A_W|$). Musimy stworzyć słownik, który zawiera $A_W$ oraz $\%$ w czasie $O(|A_W|\log |A_W|)$. Do kolejki $Q$ dodamy co najwyżej $n$ razy więc i odejmiemy co najwyżej $n$ razy. Oznacza to, że sumarycznie główna pętla zajmie $O(n\log |A_W|)$ czasu, ponieważ $n$ razy zrobimy operacje liniowe typu dodaj do kolejki, odejmij z kolejki, inkrementuj, dekrementuj, oraz $n$ razy zrobimy operacje logarytmiczne dostępu do słownika $C$.

\subsection{Algorytm pierwszy $O(n\log|A_W|+rm)$}
\subsubsection*{Algorytm}
Wykonaj algorytm permutation matching. Następnie dla każdej pozycji uznanej za dopasowanie, oblicz odległość Hamminga w czasie liniowym porównując wzorzec do tekstu od tej pozycji. Zwróć wszystkie pozycje, dla których odległość Hamminga jest $\leq k$. \subsubsection*{Poprawność}
Algorytm ten zwróci nam wszystkie pozycje będące dopasowaniem, ponieważ każde dopasowanie z $k$ błędami jest też dopasowaniem jakiejś permutacji $W$ z $k$ błędami. Algorytm permutation matching zwróci nam wszystkie te pozycje, a następnie zostaną one zwrócone przez cały algorytm, ponieważ ich odległość Hamminga będzie $\leq k$. Dodatkowo algorytm ten nie zwróci żadnej pozycji nie będącej dopasowaniem, ponieważ dla takich pozycji odległość Hamminga jest $> k$
\subsubsection*{Złożoność}
Permutation matching zajmuje $O(n\log|A_W|)$. Sprawdzenie każdego dopasowania zajmuje $O(m)$ czasu (sprawdzanie odległości Hamminga). Takich dopasowań jest $r < n$. Algorytm ten ma więc złożoność $O(n\log|A_W| +rm)$.

\subsection{Algorytm drugi $O(n\log|A_W| +dk)$}
\subsubsection{Algorytm}
Wykonaj algorytm permutation matching i oznacz wszystkie dopasowane miejsca w $\tilde{T}$.
\par
Następnie zbuduj drzewo sufiksowe dla napisu $W@\tilde{T}$, gdzie $@$ jest separatorem, oraz $@\notin A_W\cup \{\%\}$. Dla każdego dopasowanego miejsca $j$ w $\tilde{T}$ oznacz liść odpowiadający za sufiks $W@\tilde{T}$, który zaczyna się na pozycji $|\tilde{T}|-j$ od końca (czyli odpowiednik tego samego sufiksu, tylko w połączonym słowie). Dodatkowo oznacz liść odpowiadający za całe słowo $W@\tilde{T}$.
\par
Przeglądaj drzewo sufiksowe w kolejności preorder, aż napotkasz wierzchołek $u$ taki, że odpowiada za prefiks długości $\geq m$, natomiast rodzic $u$ odpowiada za prefiks długości $< m$. Następnie znajdź dowolny liść w poddrzewie $u$. Oznaczmy go $x$. Zauważmy, że wszystkie liście w tym poddrzewie mają wspólny prefiks długości $\geq m$. Więc jeśli jeden z tych liści jest oznaczony, to wszystkie są oznaczone.Jeśli prefiks długości $m$ liścia $x$ na odległość Hamminga $\leq k$, to wszystkie liście poddrzewa $u$ mają prefiksy długości $m$ z odległością Hamminga $\leq k$ (mają wspólny prefiks długości $m$). Wszystkie są więc dopasowaniem (oprócz wierzchołka odpowiadającego za $W@\tilde{T}$).
\par
Teraz zdefiniujmy funkcję $MISMATCH(j, m, k, lcp)$, która dla
\begin{itemize}
    \item indeksu $j$ w $W@\tilde{T}$
    \item $m,k$ z problemu SMK
    \item $lcp$ będącego strukturą odpowiadającą na problem longest common prefix dla drzewa sufiksowego $W@\tilde{T}$ w czasie stałym. $lcp.query(i,j)$ zwraca długość najdłuższego wspólnego prefiksu dla sufiksów $W@\tilde{T}$ zaczynających się na pozycjach $i$ oraz $j$
\end{itemize}
zwraca \textbf{true} wtw, gdy odległość Hamminga dla podsłów zaczynających się na pozycjach $1$ oraz $j$ w $W@\tilde{T}$ wynosi $\leq k$. Mając taki algorytm, dla każdego znalezionego poddrzewa $u$, po znalezieniu liścia $x$, sprawdzamy, czy $x$ jest oznaczony, oraz czy $MISMATCH(j_x,m,k,lcp)$ zwraca \textbf{true}. Jeśli tak, to znaczy, że wszystkie liście w poddrzewie $u$ (oprócz liścia odpowiadającego całemu słowu) są dopasowaniami $W$ z $k$ błędami, więc możemy zwrócić ich indeksy w $\tilde{T}$.
\begin{algorithm}[H]
\caption{MISMATCH}
\begin{algorithmic}
\INPUT{$j,m,k,lcp$}
\STATE{$w=1$\ \ \ // represents current index in pattern part of $W@\tilde{T}$}
\STATE{$t=j$\ \ \ // represents current index in text part of $W@\tilde{T}$}
\STATE{$c=0$\ \ \ // counter for mismatches}
\STATE{// until there are more than $k$ mistakes, or index in pattern is $>m$}
\STATE{// we find longest common prefix and update indexes for pattern and text}
\WHILE{$w\leq m$ \textbf{and} $c\leq k$}
    \STATE{$q:= lcp.query(w,t)$}
    \IF{$w+q\leq m$}
        \STATE{// LCP ended before end of pattern, we should increment mistake counter}
        \STATE{$c = c + 1$}
    \ENDIF
    \STATE{w = w + q + 1}
    \STATE{t = t + q + 1}
\ENDWHILE
\RETURN{$c \leq k$}
\end{algorithmic}
\end{algorithm}
\par
\subsubsection*{Poprawność}
Dlaczego oznaczamy $W@\tilde{T}$? Zauważmy, że to co robi ten algorytm, to szukanie oznaczonych liści, które mają prefiks długości $m$ z odpowiednią odległością Hamminga. Jako, że  $W@\tilde{T}$ ma prefiks długości $m$ z odległością Hamminga $0$ (część $W$), to w drzewie sufiksowym może wystąpić w tym samym poddrzewie, co faktyczne liście, które powinniśmy zliczyć. Ten prefiks nie jest jednak dopasowaniem, ponieważ nie leży w całości w $T$. Pozostałe prefiksy, które są ,,nielegalne“ (nie leżą w całości w $T$) nigdy nie wylądują w tym samym poddrzewie, ponieważ mają znak $@$, który nie występuje w $T$. Dlatego z ,,nielegalnych“ liści musimy oznaczyć liść odpowiadający $W@\tilde{T}$.
\par
Zauważmy, że dla każdego poddrzewa $u$, które rozpatrywaliśmy w algorytmie, albo wszystkie liście są oznaczone, albo wszystkie liście nie są oznaczone (mają ten sam prefiks długości $m$, więc dla wszystkich z nich permutation matching musiał znaleźć to samo). Przeglądając drzewo w kolejności preorder znajdziemy wszystkie chciane $u$ (odpowiada prefiksowi długosći $\geq m$). Dodatkowo, łatwo zauważyć, że funkcja MISMATCH poprawnie oblicza, czy odległość Hamminga takiego wspólnego prefiksu jest $\leq k$. Jeśli więc dla każdego takiego $u$ sprawdzimy dowolny liść, czy został oznaczony i wartość funkcji MISMATCH, to znajdziemy wszystkie indeksy dopasowań z co najwyżej $k$ błędami.
\subsubsection*{Złożoność}
Konstrukcja struktury drzewa sufiksowego może być zrobiona w czasie $O(n)$. Tak samo strukturę $LCP$ można utworzyć w czasie $O(n)$ \cite{GALIL198833}. Czas działania algorytmu permutation matching, to $O(n\log|A_W|)$. Przejrzenie drzewa sufiksowego w kolejności preorder zajmuje $O(n)$. Funkcję MISMATCH wykonamy $d$ razy, gdzie $d$, to upper bound na liczbę takich samych permutacji $W$ w $T$. Wywołanie funkcji $MISMATCH$ trwa $O(k)$, ponieważ każde wywołanie $lcp.query(i,j)$ jest stałe. Całość ma więc złożoność $O(n\log|A_W| + dk)$.
\begin{thebibliography}{}
\bibitem{GROSSI1989113}Grossi, R. \& Luccio, F. Simple and efficient string matching with k mismatches. {\em Information Processing Letters}. \textbf{33}, 113-120 (1989), https://www.sciencedirect.com/science/article/pii/0020019089901889
\bibitem{GALIL198833}Galil, Z. \& Giancarlo, R. Data structures and algorithms for approximate string matching. {\em Journal Of Complexity}. \textbf{4}, 33-72 (1988), https://www.sciencedirect.com/science/article/pii/0885064X88900088
\end{thebibliography}
\end{document}
