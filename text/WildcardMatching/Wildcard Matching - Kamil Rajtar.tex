\documentclass{article}
\usepackage[utf8]{inputenc}

\title{Wyszukiwanie z wildcardami w tekście i wzorcu}
\author{Kamil Rajtar}
\date{Czerwiec 2020}

\begin{document}

\newcommand\defeq{\stackrel{\mathclap{\normalfont\mbox{def}}}{=}}

\maketitle

\section*{Opis problemu}

Na wejściu do algorytmu dostajemy tekst i wzorzec w których oprócz liter mogą występować wildcardy (?) pasujące do dowolnego znaku. Na wyjściu ma się znaleźć lista pozycji w których wzorzec pasuje do tekstu.

\section*{Wersja bez wildcardów}

W rozwiązaniu problemu użyty jest splot obliczany za pomocą szybkiej transformaty Furiera (FFT) zdefiniowany następująco:

$$
    p \otimes t \defeq (\sum^{m-1}_{j=0}p_{j}t_{i+j}, 0 \leq i \leq n-m)
$$

Normalne (bez wildcardów) dopasowanie za pomocą splotu tekstu (t) i odwróconego wzorca (p) możemy obliczyć za pomocą wzoru:

$$
    \sum_{j-0}^{m-1} (p_j - t_{i+j})^2 = \sum_{j-0}^{m-1} (p^2_{j} - 2 p_{j}t_{i+j}+t^2_{i+j})
$$

Zobaczmy że lewa część równania równa 0 to znaleźliśmy dopasowanie. (Tekst nie różni się od wzorca na żadnej z m kolejnych pozycji). Prawą stronę potrafimy szybko obliczyć ponieważ podniesienie każdej pozycji do kwadratu wykonywane jest w czasie stałym a środkowy składnik liczmy za pomocą FFT w O(n log m).

\section*{Wersja z wildcardami}


Zdefiniujmy ciągi:

$$p'_j = \cases{
    0  &  p_j = '?'\cr
    1  &  wpp
}
$$
$$
t'_j = \cases{
    0  &  t_j = '?'\cr
    1  &  wpp
} $$

Wtedy łatwo jest widać że następujące równanie jest naturalnym rozszerzeniem rozwiązania poprzedniego problemu.

    $$ \sum_{j-0}^{m-1} p'_{j}t'_{i+j}(p_j - t_{i+j})^2 = 0 $$

Po rozwinięciu dostajemy formę:

$$ \sum_{j-0}^{m-1} (p'_{j}p^2_{j}t'_{i+j} - 2 p'_{j}p_{j}t_{i+j}t'_{i+j}+p'_{j}t^2_{i+j}t'_{i+j}) $$

Takie rozwinięcie łatwo jest obliczyć wykorzystując 3xFFT oraz podstawowe operacje na ciągach.

\section*{Złożoność}

Analiza złożoności jest trywialna ponieważ algorytm działa w czasie FFT.

\end{document}
